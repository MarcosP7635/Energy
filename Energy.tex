\documentclass[11pt]{article}

    \usepackage[breakable]{tcolorbox}
    \usepackage{parskip} % Stop auto-indenting (to mimic markdown behaviour)
    
    \usepackage{iftex}
    \ifPDFTeX
    	\usepackage[T1]{fontenc}
    	\usepackage{mathpazo}
    \else
    	\usepackage{fontspec}
    \fi

    % Basic figure setup, for now with no caption control since it's done
    % automatically by Pandoc (which extracts ![](path) syntax from Markdown).
    \usepackage{graphicx}
    % Maintain compatibility with old templates. Remove in nbconvert 6.0
    \let\Oldincludegraphics\includegraphics
    % Ensure that by default, figures have no caption (until we provide a
    % proper Figure object with a Caption API and a way to capture that
    % in the conversion process - todo).
    \usepackage{caption}
    \DeclareCaptionFormat{nocaption}{}
    \captionsetup{format=nocaption,aboveskip=0pt,belowskip=0pt}

    \usepackage[Export]{adjustbox} % Used to constrain images to a maximum size
    \adjustboxset{max size={0.9\linewidth}{0.9\paperheight}}
    \usepackage{float}
    \floatplacement{figure}{H} % forces figures to be placed at the correct location
    \usepackage{xcolor} % Allow colors to be defined
    \usepackage{enumerate} % Needed for markdown enumerations to work
    \usepackage{geometry} % Used to adjust the document margins
    \usepackage{amsmath} % Equations
    \usepackage{amssymb} % Equations
    \usepackage{textcomp} % defines textquotesingle
    % Hack from http://tex.stackexchange.com/a/47451/13684:
    \AtBeginDocument{%
        \def\PYZsq{\textquotesingle}% Upright quotes in Pygmentized code
    }
    \usepackage{upquote} % Upright quotes for verbatim code
    \usepackage{eurosym} % defines \euro
    \usepackage[mathletters]{ucs} % Extended unicode (utf-8) support
    \usepackage{fancyvrb} % verbatim replacement that allows latex
    \usepackage{grffile} % extends the file name processing of package graphics 
                         % to support a larger range
    \makeatletter % fix for grffile with XeLaTeX
    \def\Gread@@xetex#1{%
      \IfFileExists{"\Gin@base".bb}%
      {\Gread@eps{\Gin@base.bb}}%
      {\Gread@@xetex@aux#1}%
    }
    \makeatother

    % The hyperref package gives us a pdf with properly built
    % internal navigation ('pdf bookmarks' for the table of contents,
    % internal cross-reference links, web links for URLs, etc.)
    \usepackage{hyperref}
    % The default LaTeX title has an obnoxious amount of whitespace. By default,
    % titling removes some of it. It also provides customization options.
    \usepackage{titling}
    \usepackage{longtable} % longtable support required by pandoc >1.10
    \usepackage{booktabs}  % table support for pandoc > 1.12.2
    \usepackage[inline]{enumitem} % IRkernel/repr support (it uses the enumerate* environment)
    \usepackage[normalem]{ulem} % ulem is needed to support strikethroughs (\sout)
                                % normalem makes italics be italics, not underlines
    \usepackage{mathrsfs}
    

    
    % Colors for the hyperref package
    \definecolor{urlcolor}{rgb}{0,.145,.698}
    \definecolor{linkcolor}{rgb}{.71,0.21,0.01}
    \definecolor{citecolor}{rgb}{.12,.54,.11}

    % ANSI colors
    \definecolor{ansi-black}{HTML}{3E424D}
    \definecolor{ansi-black-intense}{HTML}{282C36}
    \definecolor{ansi-red}{HTML}{E75C58}
    \definecolor{ansi-red-intense}{HTML}{B22B31}
    \definecolor{ansi-green}{HTML}{00A250}
    \definecolor{ansi-green-intense}{HTML}{007427}
    \definecolor{ansi-yellow}{HTML}{DDB62B}
    \definecolor{ansi-yellow-intense}{HTML}{B27D12}
    \definecolor{ansi-blue}{HTML}{208FFB}
    \definecolor{ansi-blue-intense}{HTML}{0065CA}
    \definecolor{ansi-magenta}{HTML}{D160C4}
    \definecolor{ansi-magenta-intense}{HTML}{A03196}
    \definecolor{ansi-cyan}{HTML}{60C6C8}
    \definecolor{ansi-cyan-intense}{HTML}{258F8F}
    \definecolor{ansi-white}{HTML}{C5C1B4}
    \definecolor{ansi-white-intense}{HTML}{A1A6B2}
    \definecolor{ansi-default-inverse-fg}{HTML}{FFFFFF}
    \definecolor{ansi-default-inverse-bg}{HTML}{000000}

    % commands and environments needed by pandoc snippets
    % extracted from the output of `pandoc -s`
    \providecommand{\tightlist}{%
      \setlength{\itemsep}{0pt}\setlength{\parskip}{0pt}}
    \DefineVerbatimEnvironment{Highlighting}{Verbatim}{commandchars=\\\{\}}
    % Add ',fontsize=\small' for more characters per line
    \newenvironment{Shaded}{}{}
    \newcommand{\KeywordTok}[1]{\textcolor[rgb]{0.00,0.44,0.13}{\textbf{{#1}}}}
    \newcommand{\DataTypeTok}[1]{\textcolor[rgb]{0.56,0.13,0.00}{{#1}}}
    \newcommand{\DecValTok}[1]{\textcolor[rgb]{0.25,0.63,0.44}{{#1}}}
    \newcommand{\BaseNTok}[1]{\textcolor[rgb]{0.25,0.63,0.44}{{#1}}}
    \newcommand{\FloatTok}[1]{\textcolor[rgb]{0.25,0.63,0.44}{{#1}}}
    \newcommand{\CharTok}[1]{\textcolor[rgb]{0.25,0.44,0.63}{{#1}}}
    \newcommand{\StringTok}[1]{\textcolor[rgb]{0.25,0.44,0.63}{{#1}}}
    \newcommand{\CommentTok}[1]{\textcolor[rgb]{0.38,0.63,0.69}{\textit{{#1}}}}
    \newcommand{\OtherTok}[1]{\textcolor[rgb]{0.00,0.44,0.13}{{#1}}}
    \newcommand{\AlertTok}[1]{\textcolor[rgb]{1.00,0.00,0.00}{\textbf{{#1}}}}
    \newcommand{\FunctionTok}[1]{\textcolor[rgb]{0.02,0.16,0.49}{{#1}}}
    \newcommand{\RegionMarkerTok}[1]{{#1}}
    \newcommand{\ErrorTok}[1]{\textcolor[rgb]{1.00,0.00,0.00}{\textbf{{#1}}}}
    \newcommand{\NormalTok}[1]{{#1}}
    
    % Additional commands for more recent versions of Pandoc
    \newcommand{\ConstantTok}[1]{\textcolor[rgb]{0.53,0.00,0.00}{{#1}}}
    \newcommand{\SpecialCharTok}[1]{\textcolor[rgb]{0.25,0.44,0.63}{{#1}}}
    \newcommand{\VerbatimStringTok}[1]{\textcolor[rgb]{0.25,0.44,0.63}{{#1}}}
    \newcommand{\SpecialStringTok}[1]{\textcolor[rgb]{0.73,0.40,0.53}{{#1}}}
    \newcommand{\ImportTok}[1]{{#1}}
    \newcommand{\DocumentationTok}[1]{\textcolor[rgb]{0.73,0.13,0.13}{\textit{{#1}}}}
    \newcommand{\AnnotationTok}[1]{\textcolor[rgb]{0.38,0.63,0.69}{\textbf{\textit{{#1}}}}}
    \newcommand{\CommentVarTok}[1]{\textcolor[rgb]{0.38,0.63,0.69}{\textbf{\textit{{#1}}}}}
    \newcommand{\VariableTok}[1]{\textcolor[rgb]{0.10,0.09,0.49}{{#1}}}
    \newcommand{\ControlFlowTok}[1]{\textcolor[rgb]{0.00,0.44,0.13}{\textbf{{#1}}}}
    \newcommand{\OperatorTok}[1]{\textcolor[rgb]{0.40,0.40,0.40}{{#1}}}
    \newcommand{\BuiltInTok}[1]{{#1}}
    \newcommand{\ExtensionTok}[1]{{#1}}
    \newcommand{\PreprocessorTok}[1]{\textcolor[rgb]{0.74,0.48,0.00}{{#1}}}
    \newcommand{\AttributeTok}[1]{\textcolor[rgb]{0.49,0.56,0.16}{{#1}}}
    \newcommand{\InformationTok}[1]{\textcolor[rgb]{0.38,0.63,0.69}{\textbf{\textit{{#1}}}}}
    \newcommand{\WarningTok}[1]{\textcolor[rgb]{0.38,0.63,0.69}{\textbf{\textit{{#1}}}}}
    
    
    % Define a nice break command that doesn't care if a line doesn't already
    % exist.
    \def\br{\hspace*{\fill} \\* }
    % Math Jax compatibility definitions
    \def\gt{>}
    \def\lt{<}
    \let\Oldtex\TeX
    \let\Oldlatex\LaTeX
    \renewcommand{\TeX}{\textrm{\Oldtex}}
    \renewcommand{\LaTeX}{\textrm{\Oldlatex}}
    % Document parameters
    % Document title
    \title{Energy}
    
    
    
    
    
% Pygments definitions
\makeatletter
\def\PY@reset{\let\PY@it=\relax \let\PY@bf=\relax%
    \let\PY@ul=\relax \let\PY@tc=\relax%
    \let\PY@bc=\relax \let\PY@ff=\relax}
\def\PY@tok#1{\csname PY@tok@#1\endcsname}
\def\PY@toks#1+{\ifx\relax#1\empty\else%
    \PY@tok{#1}\expandafter\PY@toks\fi}
\def\PY@do#1{\PY@bc{\PY@tc{\PY@ul{%
    \PY@it{\PY@bf{\PY@ff{#1}}}}}}}
\def\PY#1#2{\PY@reset\PY@toks#1+\relax+\PY@do{#2}}

\expandafter\def\csname PY@tok@w\endcsname{\def\PY@tc##1{\textcolor[rgb]{0.73,0.73,0.73}{##1}}}
\expandafter\def\csname PY@tok@c\endcsname{\let\PY@it=\textit\def\PY@tc##1{\textcolor[rgb]{0.25,0.50,0.50}{##1}}}
\expandafter\def\csname PY@tok@cp\endcsname{\def\PY@tc##1{\textcolor[rgb]{0.74,0.48,0.00}{##1}}}
\expandafter\def\csname PY@tok@k\endcsname{\let\PY@bf=\textbf\def\PY@tc##1{\textcolor[rgb]{0.00,0.50,0.00}{##1}}}
\expandafter\def\csname PY@tok@kp\endcsname{\def\PY@tc##1{\textcolor[rgb]{0.00,0.50,0.00}{##1}}}
\expandafter\def\csname PY@tok@kt\endcsname{\def\PY@tc##1{\textcolor[rgb]{0.69,0.00,0.25}{##1}}}
\expandafter\def\csname PY@tok@o\endcsname{\def\PY@tc##1{\textcolor[rgb]{0.40,0.40,0.40}{##1}}}
\expandafter\def\csname PY@tok@ow\endcsname{\let\PY@bf=\textbf\def\PY@tc##1{\textcolor[rgb]{0.67,0.13,1.00}{##1}}}
\expandafter\def\csname PY@tok@nb\endcsname{\def\PY@tc##1{\textcolor[rgb]{0.00,0.50,0.00}{##1}}}
\expandafter\def\csname PY@tok@nf\endcsname{\def\PY@tc##1{\textcolor[rgb]{0.00,0.00,1.00}{##1}}}
\expandafter\def\csname PY@tok@nc\endcsname{\let\PY@bf=\textbf\def\PY@tc##1{\textcolor[rgb]{0.00,0.00,1.00}{##1}}}
\expandafter\def\csname PY@tok@nn\endcsname{\let\PY@bf=\textbf\def\PY@tc##1{\textcolor[rgb]{0.00,0.00,1.00}{##1}}}
\expandafter\def\csname PY@tok@ne\endcsname{\let\PY@bf=\textbf\def\PY@tc##1{\textcolor[rgb]{0.82,0.25,0.23}{##1}}}
\expandafter\def\csname PY@tok@nv\endcsname{\def\PY@tc##1{\textcolor[rgb]{0.10,0.09,0.49}{##1}}}
\expandafter\def\csname PY@tok@no\endcsname{\def\PY@tc##1{\textcolor[rgb]{0.53,0.00,0.00}{##1}}}
\expandafter\def\csname PY@tok@nl\endcsname{\def\PY@tc##1{\textcolor[rgb]{0.63,0.63,0.00}{##1}}}
\expandafter\def\csname PY@tok@ni\endcsname{\let\PY@bf=\textbf\def\PY@tc##1{\textcolor[rgb]{0.60,0.60,0.60}{##1}}}
\expandafter\def\csname PY@tok@na\endcsname{\def\PY@tc##1{\textcolor[rgb]{0.49,0.56,0.16}{##1}}}
\expandafter\def\csname PY@tok@nt\endcsname{\let\PY@bf=\textbf\def\PY@tc##1{\textcolor[rgb]{0.00,0.50,0.00}{##1}}}
\expandafter\def\csname PY@tok@nd\endcsname{\def\PY@tc##1{\textcolor[rgb]{0.67,0.13,1.00}{##1}}}
\expandafter\def\csname PY@tok@s\endcsname{\def\PY@tc##1{\textcolor[rgb]{0.73,0.13,0.13}{##1}}}
\expandafter\def\csname PY@tok@sd\endcsname{\let\PY@it=\textit\def\PY@tc##1{\textcolor[rgb]{0.73,0.13,0.13}{##1}}}
\expandafter\def\csname PY@tok@si\endcsname{\let\PY@bf=\textbf\def\PY@tc##1{\textcolor[rgb]{0.73,0.40,0.53}{##1}}}
\expandafter\def\csname PY@tok@se\endcsname{\let\PY@bf=\textbf\def\PY@tc##1{\textcolor[rgb]{0.73,0.40,0.13}{##1}}}
\expandafter\def\csname PY@tok@sr\endcsname{\def\PY@tc##1{\textcolor[rgb]{0.73,0.40,0.53}{##1}}}
\expandafter\def\csname PY@tok@ss\endcsname{\def\PY@tc##1{\textcolor[rgb]{0.10,0.09,0.49}{##1}}}
\expandafter\def\csname PY@tok@sx\endcsname{\def\PY@tc##1{\textcolor[rgb]{0.00,0.50,0.00}{##1}}}
\expandafter\def\csname PY@tok@m\endcsname{\def\PY@tc##1{\textcolor[rgb]{0.40,0.40,0.40}{##1}}}
\expandafter\def\csname PY@tok@gh\endcsname{\let\PY@bf=\textbf\def\PY@tc##1{\textcolor[rgb]{0.00,0.00,0.50}{##1}}}
\expandafter\def\csname PY@tok@gu\endcsname{\let\PY@bf=\textbf\def\PY@tc##1{\textcolor[rgb]{0.50,0.00,0.50}{##1}}}
\expandafter\def\csname PY@tok@gd\endcsname{\def\PY@tc##1{\textcolor[rgb]{0.63,0.00,0.00}{##1}}}
\expandafter\def\csname PY@tok@gi\endcsname{\def\PY@tc##1{\textcolor[rgb]{0.00,0.63,0.00}{##1}}}
\expandafter\def\csname PY@tok@gr\endcsname{\def\PY@tc##1{\textcolor[rgb]{1.00,0.00,0.00}{##1}}}
\expandafter\def\csname PY@tok@ge\endcsname{\let\PY@it=\textit}
\expandafter\def\csname PY@tok@gs\endcsname{\let\PY@bf=\textbf}
\expandafter\def\csname PY@tok@gp\endcsname{\let\PY@bf=\textbf\def\PY@tc##1{\textcolor[rgb]{0.00,0.00,0.50}{##1}}}
\expandafter\def\csname PY@tok@go\endcsname{\def\PY@tc##1{\textcolor[rgb]{0.53,0.53,0.53}{##1}}}
\expandafter\def\csname PY@tok@gt\endcsname{\def\PY@tc##1{\textcolor[rgb]{0.00,0.27,0.87}{##1}}}
\expandafter\def\csname PY@tok@err\endcsname{\def\PY@bc##1{\setlength{\fboxsep}{0pt}\fcolorbox[rgb]{1.00,0.00,0.00}{1,1,1}{\strut ##1}}}
\expandafter\def\csname PY@tok@kc\endcsname{\let\PY@bf=\textbf\def\PY@tc##1{\textcolor[rgb]{0.00,0.50,0.00}{##1}}}
\expandafter\def\csname PY@tok@kd\endcsname{\let\PY@bf=\textbf\def\PY@tc##1{\textcolor[rgb]{0.00,0.50,0.00}{##1}}}
\expandafter\def\csname PY@tok@kn\endcsname{\let\PY@bf=\textbf\def\PY@tc##1{\textcolor[rgb]{0.00,0.50,0.00}{##1}}}
\expandafter\def\csname PY@tok@kr\endcsname{\let\PY@bf=\textbf\def\PY@tc##1{\textcolor[rgb]{0.00,0.50,0.00}{##1}}}
\expandafter\def\csname PY@tok@bp\endcsname{\def\PY@tc##1{\textcolor[rgb]{0.00,0.50,0.00}{##1}}}
\expandafter\def\csname PY@tok@fm\endcsname{\def\PY@tc##1{\textcolor[rgb]{0.00,0.00,1.00}{##1}}}
\expandafter\def\csname PY@tok@vc\endcsname{\def\PY@tc##1{\textcolor[rgb]{0.10,0.09,0.49}{##1}}}
\expandafter\def\csname PY@tok@vg\endcsname{\def\PY@tc##1{\textcolor[rgb]{0.10,0.09,0.49}{##1}}}
\expandafter\def\csname PY@tok@vi\endcsname{\def\PY@tc##1{\textcolor[rgb]{0.10,0.09,0.49}{##1}}}
\expandafter\def\csname PY@tok@vm\endcsname{\def\PY@tc##1{\textcolor[rgb]{0.10,0.09,0.49}{##1}}}
\expandafter\def\csname PY@tok@sa\endcsname{\def\PY@tc##1{\textcolor[rgb]{0.73,0.13,0.13}{##1}}}
\expandafter\def\csname PY@tok@sb\endcsname{\def\PY@tc##1{\textcolor[rgb]{0.73,0.13,0.13}{##1}}}
\expandafter\def\csname PY@tok@sc\endcsname{\def\PY@tc##1{\textcolor[rgb]{0.73,0.13,0.13}{##1}}}
\expandafter\def\csname PY@tok@dl\endcsname{\def\PY@tc##1{\textcolor[rgb]{0.73,0.13,0.13}{##1}}}
\expandafter\def\csname PY@tok@s2\endcsname{\def\PY@tc##1{\textcolor[rgb]{0.73,0.13,0.13}{##1}}}
\expandafter\def\csname PY@tok@sh\endcsname{\def\PY@tc##1{\textcolor[rgb]{0.73,0.13,0.13}{##1}}}
\expandafter\def\csname PY@tok@s1\endcsname{\def\PY@tc##1{\textcolor[rgb]{0.73,0.13,0.13}{##1}}}
\expandafter\def\csname PY@tok@mb\endcsname{\def\PY@tc##1{\textcolor[rgb]{0.40,0.40,0.40}{##1}}}
\expandafter\def\csname PY@tok@mf\endcsname{\def\PY@tc##1{\textcolor[rgb]{0.40,0.40,0.40}{##1}}}
\expandafter\def\csname PY@tok@mh\endcsname{\def\PY@tc##1{\textcolor[rgb]{0.40,0.40,0.40}{##1}}}
\expandafter\def\csname PY@tok@mi\endcsname{\def\PY@tc##1{\textcolor[rgb]{0.40,0.40,0.40}{##1}}}
\expandafter\def\csname PY@tok@il\endcsname{\def\PY@tc##1{\textcolor[rgb]{0.40,0.40,0.40}{##1}}}
\expandafter\def\csname PY@tok@mo\endcsname{\def\PY@tc##1{\textcolor[rgb]{0.40,0.40,0.40}{##1}}}
\expandafter\def\csname PY@tok@ch\endcsname{\let\PY@it=\textit\def\PY@tc##1{\textcolor[rgb]{0.25,0.50,0.50}{##1}}}
\expandafter\def\csname PY@tok@cm\endcsname{\let\PY@it=\textit\def\PY@tc##1{\textcolor[rgb]{0.25,0.50,0.50}{##1}}}
\expandafter\def\csname PY@tok@cpf\endcsname{\let\PY@it=\textit\def\PY@tc##1{\textcolor[rgb]{0.25,0.50,0.50}{##1}}}
\expandafter\def\csname PY@tok@c1\endcsname{\let\PY@it=\textit\def\PY@tc##1{\textcolor[rgb]{0.25,0.50,0.50}{##1}}}
\expandafter\def\csname PY@tok@cs\endcsname{\let\PY@it=\textit\def\PY@tc##1{\textcolor[rgb]{0.25,0.50,0.50}{##1}}}

\def\PYZbs{\char`\\}
\def\PYZus{\char`\_}
\def\PYZob{\char`\{}
\def\PYZcb{\char`\}}
\def\PYZca{\char`\^}
\def\PYZam{\char`\&}
\def\PYZlt{\char`\<}
\def\PYZgt{\char`\>}
\def\PYZsh{\char`\#}
\def\PYZpc{\char`\%}
\def\PYZdl{\char`\$}
\def\PYZhy{\char`\-}
\def\PYZsq{\char`\'}
\def\PYZdq{\char`\"}
\def\PYZti{\char`\~}
% for compatibility with earlier versions
\def\PYZat{@}
\def\PYZlb{[}
\def\PYZrb{]}
\makeatother


    % For linebreaks inside Verbatim environment from package fancyvrb. 
    \makeatletter
        \newbox\Wrappedcontinuationbox 
        \newbox\Wrappedvisiblespacebox 
        \newcommand*\Wrappedvisiblespace {\textcolor{red}{\textvisiblespace}} 
        \newcommand*\Wrappedcontinuationsymbol {\textcolor{red}{\llap{\tiny$\m@th\hookrightarrow$}}} 
        \newcommand*\Wrappedcontinuationindent {3ex } 
        \newcommand*\Wrappedafterbreak {\kern\Wrappedcontinuationindent\copy\Wrappedcontinuationbox} 
        % Take advantage of the already applied Pygments mark-up to insert 
        % potential linebreaks for TeX processing. 
        %        {, <, #, %, $, ' and ": go to next line. 
        %        _, }, ^, &, >, - and ~: stay at end of broken line. 
        % Use of \textquotesingle for straight quote. 
        \newcommand*\Wrappedbreaksatspecials {% 
            \def\PYGZus{\discretionary{\char`\_}{\Wrappedafterbreak}{\char`\_}}% 
            \def\PYGZob{\discretionary{}{\Wrappedafterbreak\char`\{}{\char`\{}}% 
            \def\PYGZcb{\discretionary{\char`\}}{\Wrappedafterbreak}{\char`\}}}% 
            \def\PYGZca{\discretionary{\char`\^}{\Wrappedafterbreak}{\char`\^}}% 
            \def\PYGZam{\discretionary{\char`\&}{\Wrappedafterbreak}{\char`\&}}% 
            \def\PYGZlt{\discretionary{}{\Wrappedafterbreak\char`\<}{\char`\<}}% 
            \def\PYGZgt{\discretionary{\char`\>}{\Wrappedafterbreak}{\char`\>}}% 
            \def\PYGZsh{\discretionary{}{\Wrappedafterbreak\char`\#}{\char`\#}}% 
            \def\PYGZpc{\discretionary{}{\Wrappedafterbreak\char`\%}{\char`\%}}% 
            \def\PYGZdl{\discretionary{}{\Wrappedafterbreak\char`\$}{\char`\$}}% 
            \def\PYGZhy{\discretionary{\char`\-}{\Wrappedafterbreak}{\char`\-}}% 
            \def\PYGZsq{\discretionary{}{\Wrappedafterbreak\textquotesingle}{\textquotesingle}}% 
            \def\PYGZdq{\discretionary{}{\Wrappedafterbreak\char`\"}{\char`\"}}% 
            \def\PYGZti{\discretionary{\char`\~}{\Wrappedafterbreak}{\char`\~}}% 
        } 
        % Some characters . , ; ? ! / are not pygmentized. 
        % This macro makes them "active" and they will insert potential linebreaks 
        \newcommand*\Wrappedbreaksatpunct {% 
            \lccode`\~`\.\lowercase{\def~}{\discretionary{\hbox{\char`\.}}{\Wrappedafterbreak}{\hbox{\char`\.}}}% 
            \lccode`\~`\,\lowercase{\def~}{\discretionary{\hbox{\char`\,}}{\Wrappedafterbreak}{\hbox{\char`\,}}}% 
            \lccode`\~`\;\lowercase{\def~}{\discretionary{\hbox{\char`\;}}{\Wrappedafterbreak}{\hbox{\char`\;}}}% 
            \lccode`\~`\:\lowercase{\def~}{\discretionary{\hbox{\char`\:}}{\Wrappedafterbreak}{\hbox{\char`\:}}}% 
            \lccode`\~`\?\lowercase{\def~}{\discretionary{\hbox{\char`\?}}{\Wrappedafterbreak}{\hbox{\char`\?}}}% 
            \lccode`\~`\!\lowercase{\def~}{\discretionary{\hbox{\char`\!}}{\Wrappedafterbreak}{\hbox{\char`\!}}}% 
            \lccode`\~`\/\lowercase{\def~}{\discretionary{\hbox{\char`\/}}{\Wrappedafterbreak}{\hbox{\char`\/}}}% 
            \catcode`\.\active
            \catcode`\,\active 
            \catcode`\;\active
            \catcode`\:\active
            \catcode`\?\active
            \catcode`\!\active
            \catcode`\/\active 
            \lccode`\~`\~ 	
        }
    \makeatother

    \let\OriginalVerbatim=\Verbatim
    \makeatletter
    \renewcommand{\Verbatim}[1][1]{%
        %\parskip\z@skip
        \sbox\Wrappedcontinuationbox {\Wrappedcontinuationsymbol}%
        \sbox\Wrappedvisiblespacebox {\FV@SetupFont\Wrappedvisiblespace}%
        \def\FancyVerbFormatLine ##1{\hsize\linewidth
            \vtop{\raggedright\hyphenpenalty\z@\exhyphenpenalty\z@
                \doublehyphendemerits\z@\finalhyphendemerits\z@
                \strut ##1\strut}%
        }%
        % If the linebreak is at a space, the latter will be displayed as visible
        % space at end of first line, and a continuation symbol starts next line.
        % Stretch/shrink are however usually zero for typewriter font.
        \def\FV@Space {%
            \nobreak\hskip\z@ plus\fontdimen3\font minus\fontdimen4\font
            \discretionary{\copy\Wrappedvisiblespacebox}{\Wrappedafterbreak}
            {\kern\fontdimen2\font}%
        }%
        
        % Allow breaks at special characters using \PYG... macros.
        \Wrappedbreaksatspecials
        % Breaks at punctuation characters . , ; ? ! and / need catcode=\active 	
        \OriginalVerbatim[#1,codes*=\Wrappedbreaksatpunct]%
    }
    \makeatother

    % Exact colors from NB
    \definecolor{incolor}{HTML}{303F9F}
    \definecolor{outcolor}{HTML}{D84315}
    \definecolor{cellborder}{HTML}{CFCFCF}
    \definecolor{cellbackground}{HTML}{F7F7F7}
    
    % prompt
    \makeatletter
    \newcommand{\boxspacing}{\kern\kvtcb@left@rule\kern\kvtcb@boxsep}
    \makeatother
    \newcommand{\prompt}[4]{
        \ttfamily\llap{{\color{#2}[#3]:\hspace{3pt}#4}}\vspace{-\baselineskip}
    }
    

    
    % Prevent overflowing lines due to hard-to-break entities
    \sloppy 
    % Setup hyperref package
    \hypersetup{
      breaklinks=true,  % so long urls are correctly broken across lines
      colorlinks=true,
      urlcolor=urlcolor,
      linkcolor=linkcolor,
      citecolor=citecolor,
      }
    % Slightly bigger margins than the latex defaults
    
    \geometry{verbose,tmargin=1in,bmargin=1in,lmargin=1in,rmargin=1in}
    
    

\begin{document}
    
    \maketitle
    
    

    
    \begin{tcolorbox}[breakable, size=fbox, boxrule=1pt, pad at break*=1mm,colback=cellbackground, colframe=cellborder]
\prompt{In}{incolor}{157}{\boxspacing}
\begin{Verbatim}[commandchars=\\\{\}]
\PY{k+kn}{import} \PY{n+nn}{astropy}\PY{n+nn}{.}\PY{n+nn}{units} \PY{k}{as} \PY{n+nn}{units}
\PY{k+kn}{import} \PY{n+nn}{astropy}\PY{n+nn}{.}\PY{n+nn}{constants} \PY{k}{as} \PY{n+nn}{constants}
\PY{k+kn}{import} \PY{n+nn}{matplotlib}\PY{n+nn}{.}\PY{n+nn}{pyplot} \PY{k}{as} \PY{n+nn}{plt}
\PY{k+kn}{import} \PY{n+nn}{sympy} \PY{k}{as} \PY{n+nn}{sym}
\PY{k+kn}{import} \PY{n+nn}{numpy} \PY{k}{as} \PY{n+nn}{np} 
\PY{k+kn}{import} \PY{n+nn}{pandas} \PY{k}{as} \PY{n+nn}{pd}
\PY{k+kn}{import} \PY{n+nn}{plotly}\PY{n+nn}{.}\PY{n+nn}{express} \PY{k}{as} \PY{n+nn}{px}
\PY{k+kn}{import} \PY{n+nn}{requests}
\PY{n}{a}\PY{p}{,} \PY{n}{b}\PY{p}{,} \PY{n}{c}\PY{p}{,} \PY{n}{d}\PY{p}{,} \PY{n}{e}\PY{p}{,} \PY{n}{f}\PY{p}{,} \PY{n}{g}\PY{p}{,} \PY{n}{h}\PY{p}{,} \PY{n}{i}\PY{p}{,} \PY{n}{j}\PY{p}{,} \PY{n}{k}\PY{p}{,} \PY{n}{l}\PY{p}{,} \PY{n}{m} \PY{o}{=} \PY{n}{sym}\PY{o}{.}\PY{n}{symbols}\PY{p}{(}\PY{l+s+s1}{\PYZsq{}}\PY{l+s+s1}{a b c d e f g h i j k l m}\PY{l+s+s1}{\PYZsq{}}\PY{p}{)}
\PY{n}{n}\PY{p}{,} \PY{n}{o}\PY{p}{,} \PY{n}{p}\PY{p}{,} \PY{n}{q}\PY{p}{,} \PY{n}{r}\PY{p}{,} \PY{n}{s}\PY{p}{,} \PY{n}{t}\PY{p}{,} \PY{n}{u}\PY{p}{,} \PY{n}{v}\PY{p}{,} \PY{n}{w}\PY{p}{,} \PY{n}{x}\PY{p}{,} \PY{n}{y}\PY{p}{,} \PY{n}{z} \PY{o}{=} \PY{n}{sym}\PY{o}{.}\PY{n}{symbols}\PY{p}{(}\PY{l+s+s1}{\PYZsq{}}\PY{l+s+s1}{n o p q r s t u v w x y z}\PY{l+s+s1}{\PYZsq{}}\PY{p}{)}
\PY{n}{symbol\PYZus{}list} \PY{o}{=} \PY{p}{(}\PY{n}{a}\PY{p}{,} \PY{n}{b}\PY{p}{,} \PY{n}{c}\PY{p}{,} \PY{n}{d}\PY{p}{,} \PY{n}{e}\PY{p}{,} \PY{n}{f}\PY{p}{,} \PY{n}{g}\PY{p}{,} \PY{n}{h}\PY{p}{,} \PY{n}{i}\PY{p}{,} \PY{n}{j}\PY{p}{,} \PY{n}{k}\PY{p}{,} \PY{n}{l}\PY{p}{,} \PY{n}{m}\PY{p}{,} \PY{n}{n}\PY{p}{,} \PY{n}{o}\PY{p}{,} \PY{n}{p}\PY{p}{,} \PY{n}{q}\PY{p}{,} \PY{n}{r}\PY{p}{,} \PY{n}{s}\PY{p}{,} \PY{n}{t}\PY{p}{,} \PY{n}{u}\PY{p}{,} \PY{n}{v}\PY{p}{,}
\PY{n}{w}\PY{p}{,} \PY{n}{x}\PY{p}{,} \PY{n}{y}\PY{p}{,} \PY{n}{z}\PY{p}{)}
\PY{n}{A}\PY{p}{,} \PY{n}{B}\PY{p}{,} \PY{n}{C}\PY{p}{,} \PY{n}{D}\PY{p}{,} \PY{n}{E}\PY{p}{,} \PY{n}{F}\PY{p}{,} \PY{n}{G}\PY{p}{,} \PY{n}{H}\PY{p}{,} \PY{n}{I}\PY{p}{,} \PY{n}{J}\PY{p}{,} \PY{n}{K}\PY{p}{,} \PY{n}{L}\PY{p}{,} \PY{n}{M} \PY{o}{=} \PY{n}{sym}\PY{o}{.}\PY{n}{symbols}\PY{p}{(}\PY{l+s+s1}{\PYZsq{}}\PY{l+s+s1}{A B C D E F G H I J K L M}\PY{l+s+s1}{\PYZsq{}}\PY{p}{)}
\PY{n}{N}\PY{p}{,} \PY{n}{O}\PY{p}{,} \PY{n}{P}\PY{p}{,} \PY{n}{Q}\PY{p}{,} \PY{n}{R}\PY{p}{,} \PY{n}{S}\PY{p}{,} \PY{n}{T}\PY{p}{,} \PY{n}{U}\PY{p}{,} \PY{n}{V}\PY{p}{,} \PY{n}{W}\PY{p}{,} \PY{n}{X}\PY{p}{,} \PY{n}{Y}\PY{p}{,} \PY{n}{Z} \PY{o}{=} \PY{n}{sym}\PY{o}{.}\PY{n}{symbols}\PY{p}{(}\PY{l+s+s1}{\PYZsq{}}\PY{l+s+s1}{N O P Q R S T U V W X Y Z}\PY{l+s+s1}{\PYZsq{}}\PY{p}{)}
\PY{n}{symbol\PYZus{}list} \PY{o}{=} \PY{p}{(}\PY{n}{A}\PY{p}{,} \PY{n}{B}\PY{p}{,} \PY{n}{C}\PY{p}{,} \PY{n}{D}\PY{p}{,} \PY{n}{E}\PY{p}{,} \PY{n}{F}\PY{p}{,} \PY{n}{G}\PY{p}{,} \PY{n}{H}\PY{p}{,} \PY{n}{I}\PY{p}{,} \PY{n}{J}\PY{p}{,} \PY{n}{K}\PY{p}{,} \PY{n}{L}\PY{p}{,} \PY{n}{M}\PY{p}{,} \PY{n}{N}\PY{p}{,} \PY{n}{O}\PY{p}{,} \PY{n}{P}\PY{p}{,} \PY{n}{Q}\PY{p}{,} \PY{n}{R}\PY{p}{,} \PY{n}{S}\PY{p}{,} \PY{n}{T}\PY{p}{,} \PY{n}{U}\PY{p}{,} 
               \PY{n}{V}\PY{p}{,} \PY{n}{W}\PY{p}{,} \PY{n}{X}\PY{p}{,} \PY{n}{Y}\PY{p}{,} \PY{n}{Z}\PY{p}{)}
\PY{c+c1}{\PYZsh{}\PYZpc{}matplotlib notebook \PYZsh{}incompatible with mpmath}
\end{Verbatim}
\end{tcolorbox}

    \#Import half-lifes and energy per emission from databases
\href{https://www.zotero.org/groups/4549380/batteries/collections/59RQX9TX}{Zotero
Collection} / \href{https://www-nds.iaea.org/amdc/}{Atomic Mass Data
Center (AMDC)}

    \begin{tcolorbox}[breakable, size=fbox, boxrule=1pt, pad at break*=1mm,colback=cellbackground, colframe=cellborder]
\prompt{In}{incolor}{347}{\boxspacing}
\begin{Verbatim}[commandchars=\\\{\}]
\PY{n}{url} \PY{o}{=} \PY{l+s+s2}{\PYZdq{}}\PY{l+s+s2}{https://www\PYZhy{}nds.iaea.org/amdc/ame2020/mass\PYZus{}1.mas20.txt}\PY{l+s+s2}{\PYZdq{}}
\PY{n}{response} \PY{o}{=} \PY{n}{requests}\PY{o}{.}\PY{n}{get}\PY{p}{(}\PY{n}{url}\PY{p}{)}
\PY{n}{Atomic\PYZus{}mass\PYZus{}table\PYZus{}2020} \PY{o}{=} \PY{n}{response}\PY{o}{.}\PY{n}{text}
\PY{c+c1}{\PYZsh{}Now we want to convert a string to a pandas dataframe}
\PY{n}{Atomic\PYZus{}mass\PYZus{}table\PYZus{}2020} \PY{o}{=} \PY{n+nb}{list}\PY{p}{(}\PY{n}{Atomic\PYZus{}mass\PYZus{}table\PYZus{}2020}\PY{o}{.}\PY{n}{split}\PY{p}{(}\PY{l+s+s1}{\PYZsq{}}\PY{l+s+se}{\PYZbs{}n}\PY{l+s+s1}{\PYZsq{}}\PY{p}{)}\PY{p}{)}
\PY{n}{split\PYZus{}table} \PY{o}{=} \PY{n}{Atomic\PYZus{}mass\PYZus{}table\PYZus{}2020}\PY{p}{[}\PY{l+m+mi}{36}\PY{p}{:}\PY{p}{]}

\PY{k}{def} \PY{n+nf}{clean\PYZus{}uncertainty}\PY{p}{(}\PY{n}{uncertainty}\PY{p}{)}\PY{p}{:}
    \PY{n}{uncertainty} \PY{o}{=} \PY{n}{uncertainty}\PY{o}{.}\PY{n}{replace}\PY{p}{(}\PY{l+s+s1}{\PYZsq{}}\PY{l+s+s1}{.}\PY{l+s+s1}{\PYZsq{}}\PY{p}{,} \PY{l+s+s1}{\PYZsq{}}\PY{l+s+s1}{\PYZsq{}}\PY{p}{)}
    \PY{n}{uncertainty} \PY{o}{=} \PY{n}{uncertainty}\PY{o}{.}\PY{n}{replace}\PY{p}{(}\PY{l+s+s1}{\PYZsq{}}\PY{l+s+s1}{a}\PY{l+s+s1}{\PYZsq{}}\PY{p}{,} \PY{l+s+s1}{\PYZsq{}}\PY{l+s+s1}{0}\PY{l+s+s1}{\PYZsq{}}\PY{p}{)}
    \PY{n}{uncertainty} \PY{o}{=} \PY{n}{uncertainty}\PY{o}{.}\PY{n}{replace}\PY{p}{(}\PY{l+s+s1}{\PYZsq{}}\PY{l+s+s1}{\PYZsh{}}\PY{l+s+s1}{\PYZsq{}}\PY{p}{,} \PY{l+s+s1}{\PYZsq{}}\PY{l+s+s1}{\PYZsq{}}\PY{p}{)}
    \PY{n}{uncertainty} \PY{o}{=} \PY{n+nb}{float}\PY{p}{(}\PY{l+s+s2}{\PYZdq{}}\PY{l+s+s2}{0.}\PY{l+s+s2}{\PYZdq{}} \PY{o}{+} \PY{n}{uncertainty}\PY{p}{)}
    \PY{k}{return} \PY{n}{uncertainty}

\PY{k}{def} \PY{n+nf}{clean\PYZus{}row}\PY{p}{(}\PY{n}{row}\PY{p}{)}\PY{p}{:}
    \PY{k}{while} \PY{k+kc}{True}\PY{p}{:}
        \PY{k}{try}\PY{p}{:}
            \PY{n}{row}\PY{p}{[}\PY{l+m+mi}{2}\PY{p}{]} \PY{o}{=} \PY{n+nb}{int}\PY{p}{(}\PY{n}{row}\PY{p}{[}\PY{l+m+mi}{2}\PY{p}{]}\PY{p}{)}
            \PY{n}{number} \PY{o}{=} \PY{n}{row}\PY{o}{.}\PY{n}{pop}\PY{p}{(}\PY{l+m+mi}{0}\PY{p}{)}
        \PY{k}{except}\PY{p}{:}
            \PY{n}{row}\PY{o}{.}\PY{n}{insert}\PY{p}{(}\PY{l+m+mi}{0}\PY{p}{,} \PY{n}{number}\PY{p}{)}
            \PY{k}{break}
    \PY{c+c1}{\PYZsh{}The above while loop ensures the first column is the number of neutrons}
    \PY{k}{try}\PY{p}{:} 
        \PY{n}{row}\PY{p}{[}\PY{l+m+mi}{4}\PY{p}{]} \PY{o}{=} \PY{n+nb}{float}\PY{p}{(}\PY{n}{row}\PY{p}{[}\PY{l+m+mi}{4}\PY{p}{]}\PY{p}{)} \PY{c+c1}{\PYZsh{}if this fails, we the row is valid}
        \PY{n}{row}\PY{o}{.}\PY{n}{insert}\PY{p}{(}\PY{l+m+mi}{4}\PY{p}{,} \PY{l+s+s2}{\PYZdq{}}\PY{l+s+s2}{NA}\PY{l+s+s2}{\PYZdq{}}\PY{p}{)}
    \PY{k}{except}\PY{p}{:}
        \PY{k}{pass}
    \PY{k}{try}\PY{p}{:}
        \PY{n}{row}\PY{p}{[}\PY{l+m+mi}{10}\PY{p}{]} \PY{o}{=} \PY{n}{row}\PY{p}{[}\PY{l+m+mi}{10}\PY{p}{]}\PY{o}{.}\PY{n}{replace}\PY{p}{(}\PY{l+s+s1}{\PYZsq{}}\PY{l+s+s1}{\PYZsh{}}\PY{l+s+s1}{\PYZsq{}}\PY{p}{,} \PY{l+s+s1}{\PYZsq{}}\PY{l+s+s1}{\PYZsq{}}\PY{p}{)}
        \PY{n}{row}\PY{p}{[}\PY{l+m+mi}{10}\PY{p}{]} \PY{o}{=} \PY{n+nb}{float}\PY{p}{(}\PY{n}{row}\PY{p}{[}\PY{l+m+mi}{10}\PY{p}{]}\PY{p}{)} \PY{c+c1}{\PYZsh{}This means element 9 is *}
    \PY{k}{except}\PY{p}{:}
        \PY{n}{row}\PY{o}{.}\PY{n}{insert}\PY{p}{(}\PY{l+m+mi}{11}\PY{p}{,} \PY{l+s+s2}{\PYZdq{}}\PY{l+s+s2}{NA}\PY{l+s+s2}{\PYZdq{}}\PY{p}{)}
    \PY{c+c1}{\PYZsh{}if not (len(row) == 15):}
    \PY{c+c1}{\PYZsh{}    print(row, len(row), row[9])}
    \PY{c+c1}{\PYZsh{}print(len(row), row)}
    \PY{n}{row}\PY{p}{[}\PY{l+m+mi}{12}\PY{p}{]} \PY{o}{=} \PY{n+nb}{float}\PY{p}{(}\PY{n}{row}\PY{p}{[}\PY{l+m+mi}{12}\PY{p}{]}\PY{p}{)} \PY{o}{+} \PY{n}{clean\PYZus{}uncertainty}\PY{p}{(}\PY{n}{row}\PY{p}{[}\PY{l+m+mi}{13}\PY{p}{]}\PY{p}{)}
    \PY{c+c1}{\PYZsh{}this number was formatted weirdly, so we need to clean it up}
    \PY{n}{row}\PY{o}{.}\PY{n}{pop}\PY{p}{(}\PY{l+m+mi}{13}\PY{p}{)}
    
    \PY{k}{return} \PY{n}{row}

\PY{k}{for} \PY{n}{i} \PY{o+ow}{in} \PY{n+nb}{range}\PY{p}{(}\PY{n+nb}{len}\PY{p}{(}\PY{n}{split\PYZus{}table}\PY{p}{)}\PY{p}{)}\PY{p}{:}
    \PY{k}{try}\PY{p}{:}
        \PY{n}{split\PYZus{}table}\PY{p}{[}\PY{n}{i}\PY{p}{]} \PY{o}{=} \PY{n}{clean\PYZus{}row}\PY{p}{(}\PY{n}{split\PYZus{}table}\PY{p}{[}\PY{n}{i}\PY{p}{]}\PY{o}{.}\PY{n}{split}\PY{p}{(}\PY{p}{)}\PY{p}{)}
    \PY{k}{except}\PY{p}{:}
        \PY{n+nb}{print}\PY{p}{(}\PY{n}{split\PYZus{}table}\PY{p}{[}\PY{n}{i}\PY{p}{]}\PY{o}{.}\PY{n}{split}\PY{p}{(}\PY{p}{)}\PY{p}{)}
\PY{c+c1}{\PYZsh{}We know the column names are on row 34 (0\PYZhy{}indexed)}
\PY{c+c1}{\PYZsh{}now we will make a pandas dataframe from the list of rows}
\PY{c+c1}{\PYZsh{}Annoyingly, the column names don\PYZsq{}t include the uncertainties, so we need to add them}
\PY{n}{my\PYZus{}column\PYZus{}names} \PY{o}{=} \PY{p}{[}\PY{l+s+s2}{\PYZdq{}}\PY{l+s+s2}{N}\PY{l+s+s2}{\PYZdq{}}\PY{p}{,} \PY{l+s+s2}{\PYZdq{}}\PY{l+s+s2}{Z}\PY{l+s+s2}{\PYZdq{}}\PY{p}{,} \PY{l+s+s2}{\PYZdq{}}\PY{l+s+s2}{A}\PY{l+s+s2}{\PYZdq{}}\PY{p}{,} \PY{l+s+s2}{\PYZdq{}}\PY{l+s+s2}{Elt.}\PY{l+s+s2}{\PYZdq{}}\PY{p}{,} \PY{l+s+s2}{\PYZdq{}}\PY{l+s+s2}{Orig.}\PY{l+s+s2}{\PYZdq{}}\PY{p}{,} \PY{l+s+s2}{\PYZdq{}}\PY{l+s+s2}{Mass excess (keV)}\PY{l+s+s2}{\PYZdq{}}\PY{p}{,} \PY{l+s+s2}{\PYZdq{}}\PY{l+s+s2}{Mass excess (uncertainty)}\PY{l+s+s2}{\PYZdq{}}\PY{p}{,}
 \PY{l+s+s2}{\PYZdq{}}\PY{l+s+s2}{Binding energy per nucleon (keV)}\PY{l+s+s2}{\PYZdq{}}\PY{p}{,} \PY{l+s+s2}{\PYZdq{}}\PY{l+s+s2}{Binding energy per nucleon (uncertainty)}\PY{l+s+s2}{\PYZdq{}}\PY{p}{,} 
 \PY{l+s+s2}{\PYZdq{}}\PY{l+s+s2}{Beta\PYZhy{}decay Type}\PY{l+s+s2}{\PYZdq{}}\PY{p}{,} \PY{l+s+s2}{\PYZdq{}}\PY{l+s+s2}{Beta\PYZhy{}decay energy (keV)}\PY{l+s+s2}{\PYZdq{}}\PY{p}{,} 
 \PY{l+s+s2}{\PYZdq{}}\PY{l+s+s2}{Beta\PYZhy{}decay energy (uncertainty)}\PY{l+s+s2}{\PYZdq{}}\PY{p}{,} \PY{l+s+s2}{\PYZdq{}}\PY{l+s+s2}{Atomic mass (μu)}\PY{l+s+s2}{\PYZdq{}}\PY{p}{,} 
 \PY{l+s+s2}{\PYZdq{}}\PY{l+s+s2}{Atomic mass (uncertainty)}\PY{l+s+s2}{\PYZdq{}}\PY{p}{]}
\PY{n}{Atomic\PYZus{}mass\PYZus{}table\PYZus{}2020} \PY{o}{=} \PY{n}{pd}\PY{o}{.}\PY{n}{DataFrame}\PY{p}{(}\PY{n}{split\PYZus{}table}\PY{p}{,} \PY{n}{columns} \PY{o}{=} \PY{n}{my\PYZus{}column\PYZus{}names}\PY{p}{)}
\end{Verbatim}
\end{tcolorbox}

    \begin{Verbatim}[commandchars=\\\{\}]
[]
    \end{Verbatim}

    \begin{tcolorbox}[breakable, size=fbox, boxrule=1pt, pad at break*=1mm,colback=cellbackground, colframe=cellborder]
\prompt{In}{incolor}{348}{\boxspacing}
\begin{Verbatim}[commandchars=\\\{\}]
\PY{n}{Atomic\PYZus{}mass\PYZus{}table\PYZus{}2020}
\end{Verbatim}
\end{tcolorbox}

            \begin{tcolorbox}[breakable, size=fbox, boxrule=.5pt, pad at break*=1mm, opacityfill=0]
\prompt{Out}{outcolor}{348}{\boxspacing}
\begin{Verbatim}[commandchars=\\\{\}]
         N      Z      A  Elt. Orig. Mass excess (keV)  \textbackslash{}
0        1    0.0    1.0     n    NA           8071.32
1        0    1.0    1.0     H    NA           7288.97
2        1    1.0    2.0     H    NA           13135.7
3        2    1.0    3.0     H    NA           14949.8
4        1    2.0    3.0    He    NA           14931.2
{\ldots}    {\ldots}    {\ldots}    {\ldots}   {\ldots}   {\ldots}               {\ldots}
3554   175  118.0  293.0    Og    -a           198802\#
3555   177  117.0  294.0    Ts    -a           196397\#
3556   176  118.0  294.0    Og    -a           199320\#
3557   177  118.0  295.0    Og    -a           201369\#
3558  None    NaN    NaN  None  None              None

     Mass excess (uncertainty) Binding energy per nucleon (keV)  \textbackslash{}
0                      0.00044                              0.0
1                     0.000013                              0.0
2                     0.000015                        1112.2831
3                      0.00008                        2827.2654
4                      0.00006                       2572.68044
{\ldots}                        {\ldots}                              {\ldots}
3554                      709\#                            7078\#
3555                      593\#                            7092\#
3556                      553\#                            7079\#
3557                      655\#                            7076\#
3558                      None                             None

     Binding energy per nucleon (uncertainty) Beta-decay Type  \textbackslash{}
0                                         0.0              B-
1                                         0.0              B-
2                                      0.0002              B-
3                                      0.0003              B-
4                                     0.00015              B-
{\ldots}                                       {\ldots}             {\ldots}
3554                                       2\#              B-
3555                                       2\#              B-
3556                                       2\#              B-
3557                                       2\#              B-
3558                                     None            None

     Beta-decay energy (keV) Beta-decay energy (uncertainty)  \textbackslash{}
0                    782.347                          0.0004
1                          *                              NA
2                          *                              NA
3                     18.592                         0.00006
4                     -13736                           2000\#
{\ldots}                      {\ldots}                             {\ldots}
3554                       *                              NA
3555                   -2923                            811\#
3556                       *                              NA
3557                       *                              NA
3558                    None                            None

      Atomic mass (μu) Atomic mass (uncertainty)
0             1.008665                   0.00047
1             1.007825                  0.000014
2             2.014102                  0.000015
3             3.016049                   0.00008
4             3.016029                   0.00006
{\ldots}                {\ldots}                       {\ldots}
3554        293.213423                      761\#
3555        294.210840                      637\#
3556        294.213979                      594\#
3557        295.216178                      703\#
3558               NaN                      None

[3559 rows x 14 columns]
\end{Verbatim}
\end{tcolorbox}
        
    \begin{tcolorbox}[breakable, size=fbox, boxrule=1pt, pad at break*=1mm,colback=cellbackground, colframe=cellborder]
\prompt{In}{incolor}{350}{\boxspacing}
\begin{Verbatim}[commandchars=\\\{\}]
\PY{c+c1}{\PYZsh{}Now we want to write the dataframe to a csv file}
\PY{n}{Atomic\PYZus{}mass\PYZus{}table\PYZus{}2020}\PY{o}{.}\PY{n}{to\PYZus{}csv}\PY{p}{(}\PY{l+s+s2}{\PYZdq{}}\PY{l+s+s2}{Atomic\PYZus{}mass\PYZus{}table\PYZus{}2020.csv}\PY{l+s+s2}{\PYZdq{}}\PY{p}{)}
\end{Verbatim}
\end{tcolorbox}

    \#Now to add a column for half-life

    \#Calculations for Power vs.~Half-life

    \begin{tcolorbox}[breakable, size=fbox, boxrule=1pt, pad at break*=1mm,colback=cellbackground, colframe=cellborder]
\prompt{In}{incolor}{162}{\boxspacing}
\begin{Verbatim}[commandchars=\\\{\}]
\PY{k}{def} \PY{n+nf}{half\PYZus{}life\PYZus{}to\PYZus{}energy}\PY{p}{(}\PY{n}{half\PYZus{}life}\PY{p}{,} \PY{n}{time}\PY{p}{,} \PY{n}{initial\PYZus{}mass}\PY{p}{,} \PY{n}{decay\PYZus{}type}\PY{p}{,} \PY{n}{molar\PYZus{}mass}\PY{p}{)}\PY{p}{:}
    \PY{c+c1}{\PYZsh{}we assume half life and time have consistent units (e.g. both in seconds)}
    \PY{c+c1}{\PYZsh{}all masses are in grams}
    \PY{n}{initial\PYZus{}counts} \PY{o}{=} \PY{n}{initial\PYZus{}mass} \PY{o}{*} \PY{l+m+mf}{6.0221408} \PY{o}{*} \PY{p}{(}\PY{l+m+mi}{10}\PY{o}{*}\PY{o}{*}\PY{l+m+mi}{23}\PY{p}{)} \PY{o}{/} \PY{n}{molar\PYZus{}mass} 
    \PY{n}{decay\PYZus{}count} \PY{o}{=} \PY{n}{initial\PYZus{}counts} \PY{o}{*} \PY{p}{(}\PY{l+m+mi}{1}\PY{o}{\PYZhy{}}\PY{p}{(}\PY{o}{.}\PY{l+m+mi}{5} \PY{o}{*}\PY{o}{*} \PY{p}{(}\PY{n}{time} \PY{o}{/} \PY{n}{half\PYZus{}life}\PY{p}{)}\PY{p}{)}\PY{p}{)}    
    \PY{n}{energy} \PY{o}{=} \PY{n}{decay\PYZus{}energy}\PY{p}{(}\PY{n}{decay\PYZus{}count}\PY{p}{,} \PY{n}{decay\PYZus{}type}\PY{p}{)} \PY{o}{/} \PY{l+m+mi}{2} \PY{c+c1}{\PYZsh{}conservative estimate}
    \PY{k}{return} \PY{p}{(}\PY{n}{energy}\PY{p}{)} \PY{c+c1}{\PYZsh{}joules, counts}

\PY{k}{def} \PY{n+nf}{decay\PYZus{}energy}\PY{p}{(}\PY{n}{decay\PYZus{}counts}\PY{p}{,} \PY{n}{decay\PYZus{}type}\PY{p}{)}\PY{p}{:}
  \PY{k}{if}\PY{p}{(}\PY{n}{decay\PYZus{}type} \PY{o}{==} \PY{l+s+s2}{\PYZdq{}}\PY{l+s+s2}{beta\PYZus{}minus}\PY{l+s+s2}{\PYZdq{}}\PY{p}{)}\PY{p}{:}
    \PY{k}{return} \PY{n}{decay\PYZus{}counts} \PY{o}{*} \PY{l+m+mf}{2.7237003} \PY{o}{*} \PY{p}{(}\PY{l+m+mi}{10} \PY{o}{*}\PY{o}{*} \PY{o}{\PYZhy{}}\PY{l+m+mi}{15}\PY{p}{)} \PY{c+c1}{\PYZsh{}Joules}

\PY{n}{half\PYZus{}life\PYZus{}U\PYZus{}238} \PY{o}{=} \PY{l+m+mf}{1.41}\PY{o}{*}\PY{p}{(}\PY{l+m+mi}{10} \PY{o}{*}\PY{o}{*} \PY{l+m+mi}{17}\PY{p}{)} \PY{c+c1}{\PYZsh{}seconds }
\PY{n}{half\PYZus{}life\PYZus{}CA\PYZus{}48} \PY{o}{=} \PY{l+m+mf}{6.4} \PY{o}{*} \PY{p}{(}\PY{l+m+mi}{10} \PY{o}{*}\PY{o}{*} \PY{l+m+mf}{26.5}\PY{p}{)} \PY{c+c1}{\PYZsh{}seconds}
\PY{c+c1}{\PYZsh{}around 10\PYZca{}9.5 years :)}
\PY{n}{half\PYZus{}life} \PY{o}{=} \PY{l+m+mi}{10} \PY{o}{*}\PY{o}{*} \PY{l+m+mi}{11}
\PY{n}{mass} \PY{o}{=} \PY{l+m+mi}{10} \PY{o}{*}\PY{o}{*} \PY{o}{\PYZhy{}}\PY{l+m+mi}{3} \PY{c+c1}{\PYZsh{}grams}
\PY{n}{time} \PY{o}{=} \PY{l+m+mi}{1} \PY{c+c1}{\PYZsh{}second}
\PY{n}{power} \PY{o}{=} \PY{n}{half\PYZus{}life\PYZus{}to\PYZus{}energy}\PY{p}{(}\PY{n}{half\PYZus{}life}\PY{p}{,} \PY{n}{time}\PY{p}{,} \PY{n}{mass}\PY{p}{,} \PY{l+s+s2}{\PYZdq{}}\PY{l+s+s2}{beta\PYZus{}minus}\PY{l+s+s2}{\PYZdq{}}\PY{p}{,}                             
            \PY{l+m+mi}{63}\PY{p}{)}
\PY{n}{mass}\PY{p}{,} \PY{n}{power}\PY{p}{,} \PY{n}{time} \PY{o}{=} \PY{n}{mass} \PY{o}{*} \PY{n}{units}\PY{o}{.}\PY{n}{g}\PY{p}{,} \PY{n}{power} \PY{o}{*} \PY{n}{units}\PY{o}{.}\PY{n}{W}\PY{p}{,} \PY{n}{time} \PY{o}{*} \PY{n}{units}\PY{o}{.}\PY{n}{s}
\PY{n+nb}{print}\PY{p}{(}\PY{n}{mass}\PY{p}{,} \PY{l+s+s2}{\PYZdq{}}\PY{l+s+s2}{generates}\PY{l+s+s2}{\PYZdq{}}\PY{p}{,} \PY{n}{power}\PY{p}{,} \PY{l+s+s2}{\PYZdq{}}\PY{l+s+s2}{and loses}\PY{l+s+s2}{\PYZdq{}}\PY{p}{,} \PY{n}{mass}\PY{p}{,}
      \PY{l+s+s2}{\PYZdq{}}\PY{l+s+s2}{of mass over the first}\PY{l+s+se}{\PYZbs{}n}\PY{l+s+s2}{second :).}\PY{l+s+se}{\PYZbs{}n}\PY{l+s+s2}{ Half of the energy}\PY{l+s+s2}{\PYZdq{}} \PY{o}{+}
      \PY{l+s+s2}{\PYZdq{}}\PY{l+s+s2}{ will be lost in}\PY{l+s+s2}{\PYZdq{}}\PY{p}{,} \PY{p}{(}\PY{n}{time}\PY{p}{)}\PY{o}{.}\PY{n}{to}\PY{p}{(}\PY{n}{units}\PY{o}{.}\PY{n}{year}\PY{p}{)}\PY{p}{)}
\end{Verbatim}
\end{tcolorbox}

    \begin{Verbatim}[commandchars=\\\{\}]
0.001 g generates 9.023273319236257e-08 W and loses 0.001 g of mass over the
first
second :).
 Half of the energy will be lost in 3.168808781402895e-08 yr
    \end{Verbatim}

    \begin{tcolorbox}[breakable, size=fbox, boxrule=1pt, pad at break*=1mm,colback=cellbackground, colframe=cellborder]
\prompt{In}{incolor}{163}{\boxspacing}
\begin{Verbatim}[commandchars=\\\{\}]
\PY{l+s+sd}{\PYZsq{}\PYZsq{}\PYZsq{}}
\PY{l+s+sd}{Here we will plot half\PYZhy{}life versus power generated in first second of existence.}
\PY{l+s+sd}{Based on the plot this makes, since there are about 10\PYZca{}9.5 seconds in a century,}
\PY{l+s+sd}{the ideal half life is around 10\PYZca{}10 seconds. This will mean after a century,}
\PY{l+s+sd}{it will only half of the remaining mass (and thus presumably only produce half}
\PY{l+s+sd}{of the energy indicated in this plot). }

\PY{l+s+sd}{\PYZsq{}\PYZsq{}\PYZsq{}}
\PY{k}{def} \PY{n+nf}{calc\PYZus{}half\PYZus{}power}\PY{p}{(}\PY{n}{min\PYZus{}half\PYZus{}life}\PY{p}{,} \PY{n}{max\PYZus{}half\PYZus{}life}\PY{p}{,} \PY{n}{steps}\PY{p}{,} \PY{n}{mass}\PY{p}{,} \PY{n}{molar\PYZus{}mass}\PY{p}{)}\PY{p}{:}
  \PY{n}{precise\PYZus{}half\PYZus{}lives}\PY{p}{,} \PY{n}{power\PYZus{}array} \PY{o}{=} \PY{p}{[}\PY{p}{]}\PY{p}{,} \PY{p}{[}\PY{p}{]}
  \PY{n}{time} \PY{o}{=} \PY{l+m+mi}{1}
  \PY{k}{for} \PY{n}{exponent} \PY{o+ow}{in} \PY{n}{np}\PY{o}{.}\PY{n}{linspace}\PY{p}{(}\PY{n}{min\PYZus{}half\PYZus{}life}\PY{p}{,} \PY{n}{max\PYZus{}half\PYZus{}life}\PY{p}{,} \PY{n}{steps}\PY{p}{)}\PY{p}{:}
        \PY{n}{half\PYZus{}life} \PY{o}{=} \PY{l+m+mi}{10} \PY{o}{*}\PY{o}{*} \PY{n}{exponent}
        \PY{n}{precise\PYZus{}half\PYZus{}lives}\PY{o}{.}\PY{n}{append}\PY{p}{(}\PY{n}{half\PYZus{}life}\PY{p}{)}
        \PY{n}{power} \PY{o}{=} \PY{n}{half\PYZus{}life\PYZus{}to\PYZus{}energy}\PY{p}{(}\PY{n}{half\PYZus{}life}\PY{p}{,} \PY{n}{time}\PY{p}{,} \PY{n}{mass}\PY{p}{,} \PY{l+s+s2}{\PYZdq{}}\PY{l+s+s2}{beta\PYZus{}minus}\PY{l+s+s2}{\PYZdq{}}\PY{p}{,} 
                                    \PY{n}{molar\PYZus{}mass}\PY{p}{)}
        \PY{n}{power\PYZus{}array}\PY{o}{.}\PY{n}{append}\PY{p}{(}\PY{n}{power}\PY{p}{)}
  \PY{k}{return} \PY{p}{(}\PY{n}{precise\PYZus{}half\PYZus{}lives}\PY{p}{,} \PY{n}{power\PYZus{}array}\PY{p}{)}

\PY{k}{def} \PY{n+nf}{plot\PYZus{}power\PYZus{}vs\PYZus{}half\PYZus{}life}\PY{p}{(}\PY{n}{min\PYZus{}half\PYZus{}life}\PY{p}{,} \PY{n}{max\PYZus{}half\PYZus{}life}\PY{p}{,} \PY{n}{steps}\PY{p}{,} \PY{n}{mass}\PY{p}{,} 
                            \PY{n}{molar\PYZus{}mass}\PY{p}{,} \PY{n}{point\PYZus{}size}\PY{p}{,} \PY{n}{provide\PYZus{}fit}\PY{p}{,} \PY{n}{dpi}\PY{p}{)}\PY{p}{:}
  \PY{l+s+sd}{\PYZsq{}\PYZsq{}\PYZsq{}}
\PY{l+s+sd}{  Note that the half\PYZhy{}lifes give are assumed to be a power of 10}
\PY{l+s+sd}{  \PYZsq{}\PYZsq{}\PYZsq{}}
  \PY{n}{precise\PYZus{}half\PYZus{}lives}\PY{p}{,} \PY{n}{power\PYZus{}array} \PY{o}{=} \PY{n}{calc\PYZus{}half\PYZus{}power}\PY{p}{(}\PY{n}{min\PYZus{}half\PYZus{}life}\PY{p}{,} 
                                      \PY{n}{max\PYZus{}half\PYZus{}life}\PY{p}{,} \PY{n}{steps}\PY{p}{,} \PY{n}{mass}\PY{p}{,} \PY{n}{molar\PYZus{}mass}\PY{p}{)}
  \PY{n}{plt}\PY{o}{.}\PY{n}{figure}\PY{p}{(}\PY{n}{dpi} \PY{o}{=} \PY{n}{dpi}\PY{p}{)}
  \PY{n}{plt}\PY{o}{.}\PY{n}{scatter}\PY{p}{(}\PY{n}{x} \PY{o}{=} \PY{n}{precise\PYZus{}half\PYZus{}lives}\PY{p}{,} \PY{n}{y}\PY{o}{=} \PY{n}{power\PYZus{}array}\PY{p}{,} \PY{n}{s} \PY{o}{=} \PY{n}{point\PYZus{}size}\PY{p}{)}
  \PY{n}{plt}\PY{o}{.}\PY{n}{xscale}\PY{p}{(}\PY{l+s+s2}{\PYZdq{}}\PY{l+s+s2}{log}\PY{l+s+s2}{\PYZdq{}}\PY{p}{)}\PY{p}{,} \PY{n}{plt}\PY{o}{.}\PY{n}{yscale}\PY{p}{(}\PY{l+s+s2}{\PYZdq{}}\PY{l+s+s2}{log}\PY{l+s+s2}{\PYZdq{}}\PY{p}{)}
  \PY{n}{plt}\PY{o}{.}\PY{n}{title}\PY{p}{(}\PY{l+s+s2}{\PYZdq{}}\PY{l+s+s2}{β\PYZhy{} Decay of }\PY{l+s+s2}{\PYZdq{}} \PY{o}{+} \PY{n+nb}{str}\PY{p}{(}\PY{n}{mass} \PY{o}{*} \PY{n}{units}\PY{o}{.}\PY{n}{g}\PY{p}{)} \PY{o}{+} \PY{l+s+s2}{\PYZdq{}}\PY{l+s+s2}{ of Different Isotopes}\PY{l+s+s2}{\PYZdq{}}\PY{p}{)}
  \PY{n}{plt}\PY{o}{.}\PY{n}{ylabel}\PY{p}{(}\PY{l+s+s2}{\PYZdq{}}\PY{l+s+s2}{Power in Watts}\PY{l+s+s2}{\PYZdq{}}\PY{p}{)}
  \PY{n}{plt}\PY{o}{.}\PY{n}{xlabel}\PY{p}{(}\PY{l+s+s2}{\PYZdq{}}\PY{l+s+s2}{Half Life in Seconds}\PY{l+s+s2}{\PYZdq{}}\PY{p}{)}
  \PY{n}{plt}\PY{o}{.}\PY{n}{grid}\PY{p}{(}\PY{n}{which}\PY{o}{=}\PY{l+s+s1}{\PYZsq{}}\PY{l+s+s1}{major}\PY{l+s+s1}{\PYZsq{}}\PY{p}{,} \PY{n}{color}\PY{o}{=}\PY{l+s+s1}{\PYZsq{}}\PY{l+s+s1}{black}\PY{l+s+s1}{\PYZsq{}}\PY{p}{)}
  \PY{n}{plt}\PY{o}{.}\PY{n}{grid}\PY{p}{(}\PY{n}{which}\PY{o}{=}\PY{l+s+s1}{\PYZsq{}}\PY{l+s+s1}{minor}\PY{l+s+s1}{\PYZsq{}}\PY{p}{,} \PY{n}{color}\PY{o}{=}\PY{l+s+s1}{\PYZsq{}}\PY{l+s+s1}{grey}\PY{l+s+s1}{\PYZsq{}}\PY{p}{,} \PY{n}{alpha}\PY{o}{=}\PY{l+m+mf}{0.4}\PY{p}{)}
  \PY{n}{plt}\PY{o}{.}\PY{n}{minorticks\PYZus{}on}\PY{p}{(}\PY{p}{)}
  \PY{n}{logX}\PY{p}{,} \PY{n}{logY} \PY{o}{=} \PY{n}{np}\PY{o}{.}\PY{n}{log10}\PY{p}{(}\PY{n}{precise\PYZus{}half\PYZus{}lives}\PY{p}{)}\PY{p}{,} \PY{n}{np}\PY{o}{.}\PY{n}{log10}\PY{p}{(}\PY{n}{power\PYZus{}array}\PY{p}{)}
  \PY{n}{plt}\PY{o}{.}\PY{n}{show}\PY{p}{(}\PY{p}{)}
  \PY{k}{if} \PY{n}{provide\PYZus{}fit}\PY{p}{:}
    \PY{k}{return} \PY{n}{np}\PY{o}{.}\PY{n}{polyfit}\PY{p}{(}\PY{n}{logX}\PY{p}{,} \PY{n}{logY}\PY{p}{,} \PY{l+m+mi}{1}\PY{p}{)}
\PY{n}{plot\PYZus{}power\PYZus{}vs\PYZus{}half\PYZus{}life}\PY{p}{(}\PY{l+m+mi}{0}\PY{p}{,} \PY{l+m+mi}{15}\PY{p}{,} \PY{l+m+mi}{100}\PY{p}{,} \PY{l+m+mi}{1}\PY{p}{,} \PY{l+m+mi}{118}\PY{p}{,} \PY{n}{point\PYZus{}size} \PY{o}{=} \PY{l+m+mi}{20}\PY{p}{,} 
                        \PY{n}{provide\PYZus{}fit} \PY{o}{=} \PY{k+kc}{False}\PY{p}{,} \PY{n}{dpi} \PY{o}{=} \PY{l+m+mi}{1000}\PY{p}{)}
\end{Verbatim}
\end{tcolorbox}

    \begin{center}
    \adjustimage{max size={0.9\linewidth}{0.9\paperheight}}{Images/matplotlib1.png}
    \end{center}
    { \hspace*{\fill} \\}
    
    \begin{tcolorbox}[breakable, size=fbox, boxrule=1pt, pad at break*=1mm,colback=cellbackground, colframe=cellborder]
\prompt{In}{incolor}{164}{\boxspacing}
\begin{Verbatim}[commandchars=\\\{\}]
\PY{l+s+sd}{\PYZsq{}\PYZsq{}\PYZsq{}}
\PY{l+s+sd}{Interestingly, zooming in we have a half\PYZhy{}life of 10\PYZca{}9 seconds (30 years) }
\PY{l+s+sd}{corresponds to about 1 W/g. }
\PY{l+s+sd}{Conveniently, on this log\PYZhy{}log graph, the slope is also about \PYZhy{}1. }
\PY{l+s+sd}{\PYZsq{}\PYZsq{}\PYZsq{}}
\PY{n}{plot\PYZus{}power\PYZus{}vs\PYZus{}half\PYZus{}life}\PY{p}{(}\PY{l+m+mi}{3}\PY{p}{,} \PY{l+m+mi}{7}\PY{p}{,} \PY{l+m+mi}{500}\PY{p}{,} \PY{l+m+mi}{1}\PY{p}{,} \PY{l+m+mi}{63}\PY{p}{,} \PY{l+m+mi}{2}\PY{p}{,} \PY{n}{provide\PYZus{}fit} \PY{o}{=} \PY{k+kc}{True}\PY{p}{,} \PY{n}{dpi} \PY{o}{=} \PY{l+m+mi}{10}\PY{o}{*}\PY{o}{*}\PY{l+m+mi}{3}\PY{p}{)}
\end{Verbatim}
\end{tcolorbox}

    \begin{center}
    \adjustimage{max size={0.9\linewidth}{0.9\paperheight}}{Images/matplotlib2.png}
    \end{center}
    { \hspace*{\fill} \\}
    
            \begin{tcolorbox}[breakable, size=fbox, boxrule=.5pt, pad at break*=1mm, opacityfill=0]
\prompt{Out}{outcolor}{164}{\boxspacing}
\begin{Verbatim}[commandchars=\\\{\}]
array([-0.9999807 ,  6.95525217])
\end{Verbatim}
\end{tcolorbox}
        
    \begin{tcolorbox}[breakable, size=fbox, boxrule=1pt, pad at break*=1mm,colback=cellbackground, colframe=cellborder]
\prompt{In}{incolor}{165}{\boxspacing}
\begin{Verbatim}[commandchars=\\\{\}]
\PY{n}{half\PYZus{}life\PYZus{}Si\PYZus{}31} \PY{o}{=} \PY{p}{(}\PY{l+m+mi}{157} \PY{o}{*} \PY{n}{units}\PY{o}{.}\PY{n}{minute}\PY{p}{)}\PY{o}{.}\PY{n}{to}\PY{p}{(}\PY{n}{units}\PY{o}{.}\PY{n}{s}\PY{p}{)} \PY{o}{/} \PY{n}{units}\PY{o}{.}\PY{n}{s}
\PY{n}{power\PYZus{}Si\PYZus{}31\PYZus{}per\PYZus{}gram} \PY{o}{=} \PY{n}{half\PYZus{}life\PYZus{}to\PYZus{}energy}\PY{p}{(}\PY{n}{half\PYZus{}life\PYZus{}Si\PYZus{}31}\PY{p}{,} \PY{l+m+mi}{1}\PY{p}{,} \PY{l+m+mi}{1}\PY{p}{,} \PY{l+s+s2}{\PYZdq{}}\PY{l+s+s2}{beta\PYZus{}minus}\PY{l+s+s2}{\PYZdq{}}\PY{p}{,} 
                                  \PY{l+m+mi}{31}\PY{p}{)} \PY{o}{*} \PY{n}{units}\PY{o}{.}\PY{n}{W}
\PY{n}{power\PYZus{}Si\PYZus{}31\PYZus{}per\PYZus{}gram}
\end{Verbatim}
\end{tcolorbox}
 
            
\prompt{Out}{outcolor}{165}{}
    
    $1946.6018 \; \mathrm{W}$

    

    \begin{tcolorbox}[breakable, size=fbox, boxrule=1pt, pad at break*=1mm,colback=cellbackground, colframe=cellborder]
\prompt{In}{incolor}{166}{\boxspacing}
\begin{Verbatim}[commandchars=\\\{\}]
\PY{n}{half\PYZus{}life\PYZus{}Ni\PYZus{}63} \PY{o}{=} \PY{p}{(}\PY{l+m+mf}{100.1} \PY{o}{*} \PY{n}{units}\PY{o}{.}\PY{n}{year}\PY{p}{)}\PY{o}{.}\PY{n}{to}\PY{p}{(}\PY{n}{units}\PY{o}{.}\PY{n}{s}\PY{p}{)} \PY{o}{/} \PY{n}{units}\PY{o}{.}\PY{n}{s}
\PY{n}{power\PYZus{}Ni\PYZus{}63\PYZus{}per\PYZus{}gram} \PY{o}{=} \PY{n}{half\PYZus{}life\PYZus{}to\PYZus{}energy}\PY{p}{(}\PY{n}{half\PYZus{}life\PYZus{}Ni\PYZus{}63}\PY{p}{,} \PY{l+m+mi}{1}\PY{p}{,} \PY{l+m+mi}{1}\PY{p}{,} \PY{l+s+s2}{\PYZdq{}}\PY{l+s+s2}{beta\PYZus{}minus}\PY{l+s+s2}{\PYZdq{}}\PY{p}{,} 
                                  \PY{l+m+mi}{63}\PY{p}{)} \PY{o}{*} \PY{n}{units}\PY{o}{.}\PY{n}{W}
\PY{n}{power\PYZus{}Ni\PYZus{}63\PYZus{}per\PYZus{}gram} \PY{c+c1}{\PYZsh{}200kg could power a car for 50 years :) }
\end{Verbatim}
\end{tcolorbox}
 
            
\prompt{Out}{outcolor}{166}{}
    
    $0.0028564536 \; \mathrm{W}$

    

    \#Background physics For the nuclear reaction x + X \(\to\) y + Y we
define the Q as the energy from\\
mass loss: \begin{equation}
Q = [x_{mass}+X_{mass}-y_{mass}-Y_{mass}]c^2
\end{equation} For \(\beta^-\) decay we have \begin{equation}
\begin{split}
n\to p + e^{-} + \bar{v}\\
Q = (n_{mass} - p_{mass} - e^{-}_{mass} - \bar{v}_{mass})c^2\\
Q = .782MeV - \bar{v}_{mass}c^2
\end{split}
\end{equation} Assuming a massless neutrino this simplifies to .782 MeV

    \begin{tcolorbox}[breakable, size=fbox, boxrule=1pt, pad at break*=1mm,colback=cellbackground, colframe=cellborder]
\prompt{In}{incolor}{167}{\boxspacing}
\begin{Verbatim}[commandchars=\\\{\}]
\PY{c+c1}{\PYZsh{}half\PYZus{}life\PYZus{}to\PYZus{}energy(.987 * 3.1536 * (10 ** 9), 1, 10**\PYZhy{}6)[0]}
\PY{n}{avg\PYZus{}energy\PYZus{}per\PYZus{}decay} \PY{o}{=} \PY{l+m+mi}{17} \PY{o}{*} \PY{n}{units}\PY{o}{.}\PY{n}{keV}
\PY{n}{mass\PYZus{}predecay} \PY{o}{=} \PY{n}{constants}\PY{o}{.}\PY{n}{m\PYZus{}n}
\PY{n}{mass\PYZus{}postdecay} \PY{o}{=} \PY{n}{constants}\PY{o}{.}\PY{n}{m\PYZus{}p} \PY{o}{+} \PY{n}{constants}\PY{o}{.}\PY{n}{m\PYZus{}e}
\PY{n}{Ni63\PYZus{}half\PYZus{}life} \PY{o}{=} \PY{l+m+mf}{1.02} \PY{o}{*} \PY{l+m+mf}{3.1536} \PY{o}{*} \PY{p}{(}\PY{l+m+mi}{10} \PY{o}{*}\PY{o}{*} \PY{l+m+mi}{9}\PY{p}{)} \PY{o}{*} \PY{n}{units}\PY{o}{.}\PY{n}{s} 
\PY{n}{time} \PY{o}{=} \PY{l+m+mi}{1} \PY{o}{*} \PY{n}{units}\PY{o}{.}\PY{n}{s}
\PY{n}{initial\PYZus{}mass} \PY{o}{=} \PY{p}{(}\PY{l+m+mi}{10} \PY{o}{*}\PY{o}{*} \PY{o}{\PYZhy{}}\PY{l+m+mi}{6}\PY{p}{)} \PY{o}{*} \PY{n}{units}\PY{o}{.}\PY{n}{kg}
\PY{n}{recalc\PYZus{}energy\PYZus{}per\PYZus{}decay} \PY{o}{=} \PY{p}{(}\PY{n}{mass\PYZus{}predecay} \PY{o}{\PYZhy{}} \PY{n}{mass\PYZus{}postdecay}\PY{p}{)} \PY{o}{*} \PY{p}{(}\PY{n}{constants}\PY{o}{.}\PY{n}{c} \PY{o}{*}\PY{o}{*} \PY{l+m+mi}{2}\PY{p}{)}
\PY{n}{recalc\PYZus{}energy\PYZus{}per\PYZus{}decay}\PY{o}{.}\PY{n}{to}\PY{p}{(}\PY{n}{units}\PY{o}{.}\PY{n}{keV}\PY{p}{)} \PY{o}{/} \PY{n}{avg\PYZus{}energy\PYZus{}per\PYZus{}decay}
\end{Verbatim}
\end{tcolorbox}
 
            
\prompt{Out}{outcolor}{167}{}
    
    $46.019612 \; \mathrm{}$

    

    \#Potential Supply Chain
https://drive.google.com/file/d/1Mhe\_WbmmahkeAE\_JPnbYRvphh6IAWPwn/view?usp=sharing

This is very similar to how neutrons are currently produced in particle
accelerators :)

https://drive.google.com/file/d/1wPwC2eu6CqND1JRK\_Dgwao7nJKzyh7Xj/view?usp=sharing

https://drive.google.com/file/d/1t76jd7mjup9C\_4SqJ5opnoNNZjVcNmSs/view?usp=sharing

Thermal neutrons (0.025 eV) are sufficient
(\href{https://https://en.wikipedia.org/wiki/Neutron_temperature}{neutron
temperature},
\href{https://www.zotero.org/groups/4549380/batteries/collections/CYLBRLVK/items/T7U9LM8V/collection}{Bryskin
et al.~2004. Figure 4.})

    \begin{tcolorbox}[breakable, size=fbox, boxrule=1pt, pad at break*=1mm,colback=cellbackground, colframe=cellborder]
\prompt{In}{incolor}{168}{\boxspacing}
\begin{Verbatim}[commandchars=\\\{\}]
\PY{k}{class} \PY{n+nc}{ni63\PYZus{}setup}\PY{p}{:}
  \PY{l+s+sd}{\PYZsq{}\PYZsq{}\PYZsq{}}
\PY{l+s+sd}{  This will call helper classes depending on the different steps using in the }
\PY{l+s+sd}{  production}
\PY{l+s+sd}{  \PYZsq{}\PYZsq{}\PYZsq{}}
  
  \PY{k}{def} \PY{n+nf+fm}{\PYZus{}\PYZus{}init\PYZus{}\PYZus{}}\PY{p}{(}\PY{n+nb+bp}{self}\PY{p}{,} \PY{o}{*}\PY{o}{*}\PY{n}{kwargs}\PY{p}{)}\PY{p}{:}
    \PY{n+nb+bp}{self}\PY{o}{.}\PY{n}{assign\PYZus{}attributes}\PY{p}{(}\PY{o}{*}\PY{o}{*}\PY{n}{kwargs}\PY{p}{)}
    \PY{n+nb+bp}{self}\PY{o}{.}\PY{n}{solar\PYZus{}panel\PYZus{}dict} \PY{o}{=} \PY{p}{\PYZob{}}\PY{p}{\PYZcb{}}

  \PY{k}{def} \PY{n+nf}{assign\PYZus{}attributes}\PY{p}{(}\PY{n+nb+bp}{self}\PY{p}{,} \PY{o}{*}\PY{o}{*}\PY{n}{kwargs}\PY{p}{)}\PY{p}{:}
      \PY{k}{for} \PY{n}{key} \PY{o+ow}{in} \PY{n}{kwargs}\PY{p}{:}
          \PY{n+nb}{setattr}\PY{p}{(}\PY{n+nb+bp}{self}\PY{p}{,} \PY{n}{key}\PY{p}{,} \PY{n}{kwargs}\PY{p}{[}\PY{n}{key}\PY{p}{]}\PY{p}{)}
\end{Verbatim}
\end{tcolorbox}

    \begin{tcolorbox}[breakable, size=fbox, boxrule=1pt, pad at break*=1mm,colback=cellbackground, colframe=cellborder]
\prompt{In}{incolor}{169}{\boxspacing}
\begin{Verbatim}[commandchars=\\\{\}]
\PY{k}{class} \PY{n+nc}{solar\PYZus{}panel}\PY{p}{:}
  \PY{l+s+sd}{\PYZsq{}\PYZsq{}\PYZsq{}}
\PY{l+s+sd}{  Use to create a solar\PYZus{}panel object with a given efficiency, area, and solar flux}
\PY{l+s+sd}{  \PYZsq{}\PYZsq{}\PYZsq{}} 
  \PY{k}{def} \PY{n+nf+fm}{\PYZus{}\PYZus{}init\PYZus{}\PYZus{}}\PY{p}{(}\PY{n+nb+bp}{self}\PY{p}{,} \PY{o}{*}\PY{o}{*}\PY{n}{kwargs}\PY{p}{)}\PY{p}{:}
    \PY{n+nb+bp}{self}\PY{o}{.}\PY{n}{assign\PYZus{}attributes}\PY{p}{(}\PY{o}{*}\PY{o}{*}\PY{n}{kwargs}\PY{p}{)}

  \PY{k}{def} \PY{n+nf}{assign\PYZus{}attributes}\PY{p}{(}\PY{n+nb+bp}{self}\PY{p}{,} \PY{o}{*}\PY{o}{*}\PY{n}{kwargs}\PY{p}{)}\PY{p}{:}
      \PY{k}{for} \PY{n}{key} \PY{o+ow}{in} \PY{n}{kwargs}\PY{p}{:}
          \PY{n+nb}{setattr}\PY{p}{(}\PY{n+nb+bp}{self}\PY{p}{,} \PY{n}{key}\PY{p}{,} \PY{n}{kwargs}\PY{p}{[}\PY{n}{key}\PY{p}{]}\PY{p}{)}

  \PY{k}{def} \PY{n+nf}{calc\PYZus{}voltage}\PY{p}{(}\PY{n+nb+bp}{self}\PY{p}{)}\PY{p}{:}
    \PY{l+s+sd}{\PYZsq{}\PYZsq{}\PYZsq{}}
\PY{l+s+sd}{    By definition}
\PY{l+s+sd}{    power = solar\PYZus{}flux * efficiency * area}
\PY{l+s+sd}{    power = voltage * current}
\PY{l+s+sd}{    Taking the equivalent resistance of the entire circuit, we have}
\PY{l+s+sd}{    voltage = current * resistance}
\PY{l+s+sd}{    current = voltage / resistance}
\PY{l+s+sd}{    power = voltage\PYZca{}2 / resistance}
\PY{l+s+sd}{    voltage = sqrt(power * resistance)}
\PY{l+s+sd}{    voltage = sqrt(solar\PYZus{}flux * efficiency * area * resistance)}
\PY{l+s+sd}{    \PYZsq{}\PYZsq{}\PYZsq{}} 
    \PY{n}{voltage} \PY{o}{=}  \PY{n}{np}\PY{o}{.}\PY{n}{sqrt}\PY{p}{(}\PY{n+nb+bp}{self}\PY{o}{.}\PY{n}{solar\PYZus{}flux} \PY{o}{*} \PY{n+nb+bp}{self}\PY{o}{.}\PY{n}{solar\PYZus{}panel\PYZus{}efficiency} \PY{o}{*} 
                         \PY{n+nb+bp}{self}\PY{o}{.}\PY{n}{solar\PYZus{}panel\PYZus{}area} \PY{o}{*} \PY{n+nb+bp}{self}\PY{o}{.}\PY{n}{resistance}\PY{p}{)}
    \PY{n+nb+bp}{self}\PY{o}{.}\PY{n}{voltage} \PY{o}{=} \PY{n}{voltage}\PY{o}{.}\PY{n}{to}\PY{p}{(}\PY{n}{V}\PY{p}{)}

  \PY{k}{def} \PY{n+nf}{calc\PYZus{}charge\PYZus{}plane}\PY{p}{(}\PY{n+nb+bp}{self}\PY{p}{)}\PY{p}{:}
  \PY{c+c1}{\PYZsh{}Modify appropriately for the shape of the capacitor}
    \PY{n+nb+bp}{self}\PY{o}{.}\PY{n}{charge} \PY{o}{=} \PY{n+nb+bp}{self}\PY{o}{.}\PY{n}{capacitance} \PY{o}{*} \PY{n+nb+bp}{self}\PY{o}{.}\PY{n}{voltage}

  \PY{k}{def} \PY{n+nf}{calc\PYZus{}gamma\PYZus{}ray\PYZus{}flux}\PY{p}{(}\PY{n+nb+bp}{self}\PY{p}{)}\PY{p}{:}
    \PY{n+nb+bp}{self}\PY{o}{.}\PY{n}{gamma\PYZus{}ray\PYZus{}flux} \PY{o}{=} \PY{p}{(}\PY{p}{(}\PY{n+nb+bp}{self}\PY{o}{.}\PY{n}{charge}\PY{o}{*}\PY{o}{*}\PY{l+m+mi}{2}\PY{p}{)} \PY{o}{*} \PY{p}{(}\PY{n+nb+bp}{self}\PY{o}{.}\PY{n}{charge\PYZus{}e\PYZus{}distance}\PY{o}{*}\PY{o}{*}\PY{o}{\PYZhy{}}\PY{l+m+mi}{4}\PY{p}{)} \PY{o}{*} 
                           \PY{n+nb+bp}{self}\PY{o}{.}\PY{n}{brehm\PYZus{}coeff} \PY{o}{*} \PY{n+nb+bp}{self}\PY{o}{.}\PY{n}{cathode\PYZus{}ray\PYZus{}flux}\PY{p}{)}

  \PY{k}{def} \PY{n+nf}{calc\PYZus{}ni63\PYZus{}production\PYZus{}speed}\PY{p}{(}\PY{n+nb+bp}{self}\PY{p}{)}\PY{p}{:}
    \PY{n+nb+bp}{self}\PY{o}{.}\PY{n}{ni63\PYZus{}production\PYZus{}speed} \PY{o}{=} \PY{p}{(}\PY{n+nb+bp}{self}\PY{o}{.}\PY{n}{gamma\PYZus{}ray\PYZus{}flux} \PY{o}{*} \PY{n+nb+bp}{self}\PY{o}{.}\PY{n}{donor\PYZus{}cross\PYZus{}section}
                                  \PY{o}{*} \PY{n+nb+bp}{self}\PY{o}{.}\PY{n}{target\PYZus{}cross\PYZus{}section}\PY{p}{)}
\end{Verbatim}
\end{tcolorbox}

    \begin{tcolorbox}[breakable, size=fbox, boxrule=1pt, pad at break*=1mm,colback=cellbackground, colframe=cellborder]
\prompt{In}{incolor}{170}{\boxspacing}
\begin{Verbatim}[commandchars=\\\{\}]
\PY{k}{def} \PY{n+nf}{calc\PYZus{}brehm\PYZus{}coeff}\PY{p}{(}\PY{p}{)}\PY{p}{:}
  \PY{n}{echarge} \PY{o}{=} \PY{l+m+mf}{1.6021766} \PY{o}{*} \PY{p}{(}\PY{l+m+mi}{10} \PY{o}{*}\PY{o}{*} \PY{o}{\PYZhy{}}\PY{l+m+mi}{19}\PY{p}{)} \PY{o}{*} \PY{n}{C}
  \PY{k}{return} \PY{p}{(}\PY{p}{(}\PY{n}{echarge}\PY{o}{*}\PY{o}{*}\PY{l+m+mi}{4}\PY{p}{)} \PY{o}{/} \PY{p}{(}\PY{l+m+mi}{96} \PY{o}{*} \PY{p}{(}\PY{p}{(}\PY{n}{math}\PY{o}{.}\PY{n}{pi} \PY{o}{*} \PY{n}{c} \PY{o}{*} \PY{l+m+mf}{8.8541878128} \PY{o}{*} \PY{p}{(}\PY{l+m+mi}{10}\PY{o}{*}\PY{o}{*}\PY{o}{\PYZhy{}}\PY{l+m+mi}{12}\PY{p}{)} \PY{o}{*} \PY{n}{F} \PY{o}{/} \PY{n}{m}\PY{p}{)}\PY{o}{*}\PY{o}{*}\PY{l+m+mi}{3}\PY{p}{)} \PY{o}{*} 
                    \PY{p}{(}\PY{n}{m\PYZus{}e} \PY{o}{*}\PY{o}{*} \PY{l+m+mi}{2}\PY{p}{)}\PY{p}{)}\PY{p}{)}

\PY{k}{def} \PY{n+nf}{calc\PYZus{}cathode\PYZus{}ray\PYZus{}flux}\PY{p}{(}\PY{p}{)}\PY{p}{:}
  \PY{k}{return}
\end{Verbatim}
\end{tcolorbox}

    \begin{tcolorbox}[breakable, size=fbox, boxrule=1pt, pad at break*=1mm,colback=cellbackground, colframe=cellborder]
\prompt{In}{incolor}{171}{\boxspacing}
\begin{Verbatim}[commandchars=\\\{\}]
\PY{l+s+sd}{\PYZsq{}\PYZsq{}\PYZsq{}}
\PY{l+s+sd}{https://www.thermofisher.com/order/catalog/product/1517021A }

\PY{l+s+sd}{50 W is sufficient for 10\PYZca{}8 n/s}
\PY{l+s+sd}{\PYZsq{}\PYZsq{}\PYZsq{}}
\end{Verbatim}
\end{tcolorbox}

            \begin{tcolorbox}[breakable, size=fbox, boxrule=.5pt, pad at break*=1mm, opacityfill=0]
\prompt{Out}{outcolor}{171}{\boxspacing}
\begin{Verbatim}[commandchars=\\\{\}]
'\textbackslash{}nhttps://www.thermofisher.com/order/catalog/product/1517021A \textbackslash{}n\textbackslash{}n50 W is
sufficient for 10\^{}8 n/s\textbackslash{}n'
\end{Verbatim}
\end{tcolorbox}
        
    \#Theoretical Maximum Production per m\(^2\) solar panel

    With a power input of \(P\) and a \(E_{\gamma}\) joules for each gamma
ray, assuming each gamma ray has a \(\rho\) probability of producing
\(^{63}\)Ni we have \begin{equation}
\begin{split}
N = \frac{P\rho}{E_{\gamma}}\\
\end{split}
\end{equation}

    \begin{tcolorbox}[breakable, size=fbox, boxrule=1pt, pad at break*=1mm,colback=cellbackground, colframe=cellborder]
\prompt{In}{incolor}{172}{\boxspacing}
\begin{Verbatim}[commandchars=\\\{\}]
\PY{k}{def} \PY{n+nf}{g\PYZus{}per\PYZus{}year}\PY{p}{(}\PY{n}{solar\PYZus{}flux}\PY{p}{,} \PY{n}{area}\PY{p}{,} \PY{n}{efficiency}\PY{p}{,} \PY{n}{energy\PYZus{}per\PYZus{}neutron}\PY{p}{,} \PY{n}{molar\PYZus{}mass}\PY{p}{)}\PY{p}{:}
  \PY{n}{power} \PY{o}{=} \PY{n}{solar\PYZus{}flux} \PY{o}{*} \PY{n}{area} \PY{o}{*} \PY{n}{efficiency}
  \PY{n}{nGamma} \PY{o}{=} \PY{n}{power} \PY{o}{/} \PY{n}{energy\PYZus{}per\PYZus{}neutron}
  \PY{n}{isotopes\PYZus{}per\PYZus{}second} \PY{o}{=} \PY{n}{nGamma}\PY{o}{.}\PY{n}{to}\PY{p}{(}\PY{n}{units}\PY{o}{.}\PY{n}{s} \PY{o}{*}\PY{o}{*} \PY{o}{\PYZhy{}}\PY{l+m+mi}{1}\PY{p}{)} 
  \PY{n}{molPerS} \PY{o}{=} \PY{p}{(}\PY{n}{isotopes\PYZus{}per\PYZus{}second} \PY{o}{/} \PY{p}{(}\PY{n}{constants}\PY{o}{.}\PY{n}{N\PYZus{}A}\PY{p}{)}\PY{p}{)}\PY{o}{.}\PY{n}{to}\PY{p}{(}\PY{n}{units}\PY{o}{.}\PY{n}{mol} \PY{o}{*} \PY{n}{units}\PY{o}{.}\PY{n}{s} \PY{o}{*}\PY{o}{*} \PY{o}{\PYZhy{}}\PY{l+m+mi}{1}\PY{p}{)}
  \PY{k}{return} \PY{p}{(}\PY{n}{molPerS} \PY{o}{*} \PY{p}{(}\PY{n}{molar\PYZus{}mass}\PY{p}{)}\PY{p}{)}\PY{o}{.}\PY{n}{to}\PY{p}{(}\PY{n}{units}\PY{o}{.}\PY{n}{g} \PY{o}{*} \PY{n}{units}\PY{o}{.}\PY{n}{year} \PY{o}{*}\PY{o}{*} \PY{o}{\PYZhy{}}\PY{l+m+mi}{1}\PY{p}{)}

\PY{n}{solar\PYZus{}flux} \PY{o}{=} \PY{l+m+mi}{1000} \PY{o}{*} \PY{n}{units}\PY{o}{.}\PY{n}{W} \PY{o}{/} \PY{p}{(}\PY{n}{units}\PY{o}{.}\PY{n}{m} \PY{o}{*}\PY{o}{*} \PY{l+m+mi}{2}\PY{p}{)}
\PY{n}{area} \PY{o}{=} \PY{l+m+mi}{1} \PY{o}{*} \PY{n}{units}\PY{o}{.}\PY{n}{m} \PY{o}{*}\PY{o}{*} \PY{l+m+mi}{2}
\PY{n}{efficiency} \PY{o}{=} \PY{l+m+mi}{1} \PY{c+c1}{\PYZsh{}any real world factors that affect the production rate}
\PY{n}{energy\PYZus{}per\PYZus{}neutron} \PY{o}{=} \PY{o}{.}\PY{l+m+mi}{075} \PY{o}{*} \PY{n}{units}\PY{o}{.}\PY{n}{MeV}
\PY{n}{molar\PYZus{}mass} \PY{o}{=} \PY{l+m+mi}{63} \PY{o}{*} \PY{n}{units}\PY{o}{.}\PY{n}{g} \PY{o}{/} \PY{n}{units}\PY{o}{.}\PY{n}{mol}
\PY{n}{rate} \PY{o}{=} \PY{n}{g\PYZus{}per\PYZus{}year}\PY{p}{(}\PY{n}{solar\PYZus{}flux}\PY{p}{,} \PY{n}{area}\PY{p}{,} \PY{n}{efficiency}\PY{p}{,} \PY{n}{energy\PYZus{}per\PYZus{}neutron}\PY{p}{,} \PY{n}{molar\PYZus{}mass}\PY{p}{)}
\PY{n}{rate}
\end{Verbatim}
\end{tcolorbox}
 
            
\prompt{Out}{outcolor}{172}{}
    
    $274.74004 \; \mathrm{\frac{g}{yr}}$

    

    \#\#It will take 5,000 years/m\(^2\) to make enough for a car to work
for 50 years. We have a theoretical maximum of \(10^{-6}\) mol nickel-63
per second for every square meter of sunlight collected per second = 36
moles per year \textgreater{} 2kg. Every car will need 5 years. We need
400 million m\(^2\)

    \begin{tcolorbox}[breakable, size=fbox, boxrule=1pt, pad at break*=1mm,colback=cellbackground, colframe=cellborder]
\prompt{In}{incolor}{173}{\boxspacing}
\begin{Verbatim}[commandchars=\\\{\}]
\PY{n}{earth\PYZus{}rad} \PY{o}{=} \PY{l+m+mi}{6400} \PY{o}{*} \PY{n}{units}\PY{o}{.}\PY{n}{km}
\PY{n}{earth\PYZus{}surface\PYZus{}area} \PY{o}{=} \PY{n}{np}\PY{o}{.}\PY{n}{pi} \PY{o}{*} \PY{l+m+mi}{4} \PY{o}{*} \PY{p}{(}\PY{n}{earth\PYZus{}rad} \PY{o}{*}\PY{o}{*} \PY{l+m+mi}{2}\PY{p}{)}
\PY{n}{panels\PYZus{}area} \PY{o}{=} \PY{l+m+mi}{400} \PY{o}{*} \PY{p}{(}\PY{l+m+mi}{10}\PY{o}{*}\PY{o}{*}\PY{l+m+mi}{6}\PY{p}{)} \PY{o}{*} \PY{p}{(}\PY{n}{units}\PY{o}{.}\PY{n}{m} \PY{o}{*}\PY{o}{*} \PY{l+m+mi}{2}\PY{p}{)}
\PY{n}{panels\PYZus{}area}\PY{o}{.}\PY{n}{to}\PY{p}{(}\PY{n}{units}\PY{o}{.}\PY{n}{km} \PY{o}{*}\PY{o}{*} \PY{l+m+mi}{2}\PY{p}{)} \PY{o}{/} \PY{n}{earth\PYZus{}surface\PYZus{}area}
\end{Verbatim}
\end{tcolorbox}
 
            
\prompt{Out}{outcolor}{173}{}
    
    $7.7712375 \times 10^{-7} \; \mathrm{}$

    

    \#We need to cover 1-millionth of the Earth in solar panels. More
realistically, 400 mi\(^2\)

    \#\#Caltech People who have done photoneutron work * S.R. Golwala * T.
Aralis

    \#\#What is the energy per gamma ray and probability of Ni63 production
needed to make this technology competiting in the long term compared to
current forms of energy storage?

The emission of radiation by accelerating charges is derived in Chapter
14 and 15 of the 2nd edition of Jackson's E\&M.

    \begin{tcolorbox}[breakable, size=fbox, boxrule=1pt, pad at break*=1mm,colback=cellbackground, colframe=cellborder]
\prompt{In}{incolor}{174}{\boxspacing}
\begin{Verbatim}[commandchars=\\\{\}]
\PY{l+s+sd}{\PYZsq{}\PYZsq{}\PYZsq{}}
\PY{l+s+sd}{Source: }
\PY{l+s+sd}{https://www.iea.org/reports/key\PYZhy{}world\PYZhy{}energy\PYZhy{}statistics\PYZhy{}2021/final\PYZhy{}consumption }
\PY{l+s+sd}{\PYZsq{}\PYZsq{}\PYZsq{}}
\PY{n}{annual\PYZus{}global\PYZus{}energy\PYZus{}consumed} \PY{o}{=} \PY{p}{(}\PY{l+m+mi}{450} \PY{o}{*} \PY{n}{units}\PY{o}{.}\PY{n}{EJ}\PY{p}{)}\PY{o}{.}\PY{n}{to}\PY{p}{(}\PY{n}{units}\PY{o}{.}\PY{n}{W} \PY{o}{*} \PY{n}{units}\PY{o}{.}\PY{n}{year}\PY{p}{)}
\PY{n}{global\PYZus{}power\PYZus{}consumed} \PY{o}{=} \PY{n}{annual\PYZus{}global\PYZus{}energy\PYZus{}consumed} \PY{o}{/} \PY{p}{(}\PY{l+m+mi}{1} \PY{o}{*} \PY{n}{units}\PY{o}{.}\PY{n}{year}\PY{p}{)}
\PY{n}{efficiency} \PY{o}{=} \PY{o}{.}\PY{l+m+mi}{2}
\PY{n}{power\PYZus{}per\PYZus{}area} \PY{o}{=} \PY{l+m+mi}{500} \PY{o}{*} \PY{n}{units}\PY{o}{.}\PY{n}{W} \PY{o}{/} \PY{p}{(}\PY{n}{units}\PY{o}{.}\PY{n}{m} \PY{o}{*}\PY{o}{*} \PY{l+m+mi}{2}\PY{p}{)} \PY{o}{*} \PY{n}{efficiency}
\PY{n}{area\PYZus{}needed} \PY{o}{=} \PY{p}{(}\PY{n}{global\PYZus{}power\PYZus{}consumed} \PY{o}{/} \PY{n}{power\PYZus{}per\PYZus{}area}\PY{p}{)}\PY{o}{.}\PY{n}{to}\PY{p}{(}\PY{n}{units}\PY{o}{.}\PY{n}{km} \PY{o}{*}\PY{o}{*} \PY{l+m+mi}{2}\PY{p}{)}
\PY{n}{area\PYZus{}needed}
\end{Verbatim}
\end{tcolorbox}
 
            
\prompt{Out}{outcolor}{174}{}
    
    $142596.4 \; \mathrm{km^{2}}$

    

    \begin{tcolorbox}[breakable, size=fbox, boxrule=1pt, pad at break*=1mm,colback=cellbackground, colframe=cellborder]
\prompt{In}{incolor}{175}{\boxspacing}
\begin{Verbatim}[commandchars=\\\{\}]
\PY{n}{global\PYZus{}power\PYZus{}consumed} \PY{o}{/} \PY{n}{power\PYZus{}Si\PYZus{}31\PYZus{}per\PYZus{}gram}
\end{Verbatim}
\end{tcolorbox}
 
            
\prompt{Out}{outcolor}{175}{}
    
    $7.3254015 \times 10^{9} \; \mathrm{}$

    

    \begin{tcolorbox}[breakable, size=fbox, boxrule=1pt, pad at break*=1mm,colback=cellbackground, colframe=cellborder]
\prompt{In}{incolor}{176}{\boxspacing}
\begin{Verbatim}[commandchars=\\\{\}]
\PY{n}{global\PYZus{}power\PYZus{}consumed}
\end{Verbatim}
\end{tcolorbox}
 
            
\prompt{Out}{outcolor}{176}{}
    
    $1.425964 \times 10^{13} \; \mathrm{W}$

    

    \begin{tcolorbox}[breakable, size=fbox, boxrule=1pt, pad at break*=1mm,colback=cellbackground, colframe=cellborder]
\prompt{In}{incolor}{ }{\boxspacing}
\begin{Verbatim}[commandchars=\\\{\}]

\end{Verbatim}
\end{tcolorbox}

    \begin{tcolorbox}[breakable, size=fbox, boxrule=1pt, pad at break*=1mm,colback=cellbackground, colframe=cellborder]
\prompt{In}{incolor}{177}{\boxspacing}
\begin{Verbatim}[commandchars=\\\{\}]
\PY{n}{area\PYZus{}California\PYZus{}state} \PY{o}{=} \PY{l+m+mf}{423970.694} \PY{o}{*} \PY{p}{(}\PY{n}{units}\PY{o}{.}\PY{n}{km}\PY{o}{*}\PY{o}{*}\PY{l+m+mi}{2}\PY{p}{)}
\PY{n}{area\PYZus{}California\PYZus{}state} \PY{o}{/} \PY{n}{area\PYZus{}needed}
\end{Verbatim}
\end{tcolorbox}
 
            
\prompt{Out}{outcolor}{177}{}
    
    $2.9732217 \; \mathrm{}$

    

    \#If we cover 1/3 of California in solar panels, we could power the
world.

    \#We need to cover half a million square miles with solar panels and
rapidly replant native flora.

    \begin{tcolorbox}[breakable, size=fbox, boxrule=1pt, pad at break*=1mm,colback=cellbackground, colframe=cellborder]
\prompt{In}{incolor}{178}{\boxspacing}
\begin{Verbatim}[commandchars=\\\{\}]
\PY{c+c1}{\PYZsh{}https://doi.org/10.1103/PhysRevX.7.041003}
\PY{n+nb}{input} \PY{o}{=} \PY{l+m+mi}{40} \PY{o}{*} \PY{n}{units}\PY{o}{.}\PY{n}{PW}
\PY{n}{Egamma}\PY{o}{=} \PY{l+m+mi}{2} \PY{o}{*} \PY{n}{units}\PY{o}{.}\PY{n}{MeV}
\PY{n}{max\PYZus{}photon\PYZus{}flux} \PY{o}{=} \PY{p}{(}\PY{n+nb}{input}\PY{o}{/}\PY{n}{Egamma}\PY{p}{)}\PY{o}{.}\PY{n}{to}\PY{p}{(}\PY{n}{units}\PY{o}{.}\PY{n}{s} \PY{o}{*}\PY{o}{*} \PY{o}{\PYZhy{}}\PY{l+m+mi}{1}\PY{p}{)}
\PY{n}{max\PYZus{}photon\PYZus{}flux}
\end{Verbatim}
\end{tcolorbox}
 
            
\prompt{Out}{outcolor}{178}{}
    
    $1.2483018 \times 10^{29} \; \mathrm{\frac{1}{s}}$

    

    \#\#Rewrite clas to be based on generating the right energy neutrons for
Ni63 production.

    \begin{tcolorbox}[breakable, size=fbox, boxrule=1pt, pad at break*=1mm,colback=cellbackground, colframe=cellborder]
\prompt{In}{incolor}{179}{\boxspacing}
\begin{Verbatim}[commandchars=\\\{\}]
\PY{n}{frequency} \PY{o}{=} \PY{p}{(}\PY{p}{(}\PY{l+m+mi}{2} \PY{o}{*} \PY{n}{units}\PY{o}{.}\PY{n}{MeV}\PY{p}{)}\PY{o}{.}\PY{n}{to}\PY{p}{(}\PY{n}{units}\PY{o}{.}\PY{n}{J}\PY{p}{)} \PY{o}{/} \PY{n}{constants}\PY{o}{.}\PY{n}{h}\PY{p}{)}\PY{o}{.}\PY{n}{to}\PY{p}{(}\PY{n}{units}\PY{o}{.}\PY{n}{Hz}\PY{p}{)}
\PY{n}{frequency} 
\end{Verbatim}
\end{tcolorbox}
 
            
\prompt{Out}{outcolor}{179}{}
    
    $4.8359785 \times 10^{20} \; \mathrm{Hz}$

    

    \#A system that can convert between different isotopes depending on the
power demand would be ideal.

\begin{itemize}
\tightlist
\item
  Requires a fast, compact, and energy efficient way to convert between
  emitted electrons and neutrons.
\end{itemize}

Possibilities include using the Brehmsstrahlung effect to create gamma
rays which can then be used to generate photoneutrons. This seems like
the crux of the system.

\begin{itemize}
\tightlist
\item
\end{itemize}

    

    \begin{tcolorbox}[breakable, size=fbox, boxrule=1pt, pad at break*=1mm,colback=cellbackground, colframe=cellborder]
\prompt{In}{incolor}{180}{\boxspacing}
\begin{Verbatim}[commandchars=\\\{\}]
\PY{k}{def} \PY{n+nf}{total\PYZus{}brehmsstrahlung\PYZus{}power}\PY{p}{(}\PY{n}{velocity}\PY{p}{,} \PY{n}{charge}\PY{p}{,} \PY{n}{acceleration}\PY{p}{)}\PY{p}{:}
  \PY{l+s+sd}{\PYZsq{}\PYZsq{}\PYZsq{}}
\PY{l+s+sd}{  Source: }
\PY{l+s+sd}{  https://en.wikipedia.org/wiki/Bremsstrahlung\PYZsh{}Total\PYZus{}radiated\PYZus{}power }
\PY{l+s+sd}{  \PYZsq{}\PYZsq{}\PYZsq{}}
  \PY{n}{beta} \PY{o}{=} \PY{n}{velocity} \PY{o}{/} \PY{n}{constants}\PY{o}{.}\PY{n}{c}
  \PY{n}{gamma} \PY{o}{=} \PY{p}{(}\PY{l+m+mi}{1} \PY{o}{\PYZhy{}} \PY{p}{(}\PY{n}{beta} \PY{o}{*}\PY{o}{*} \PY{l+m+mi}{2}\PY{p}{)}\PY{p}{)} \PY{o}{*}\PY{o}{*} \PY{o}{\PYZhy{}}\PY{o}{.}\PY{l+m+mi}{5}
  \PY{n}{beta\PYZus{}dot} \PY{o}{=} \PY{p}{(}\PY{n}{acceleration} \PY{o}{/} \PY{n}{constants}\PY{o}{.}\PY{n}{c}\PY{p}{)} 
  \PY{n}{beta\PYZus{}term} \PY{o}{=} \PY{p}{(}\PY{n}{beta\PYZus{}dot} \PY{o}{*}\PY{o}{*} \PY{l+m+mi}{2}\PY{p}{)} \PY{o}{+} \PY{p}{(}\PY{p}{(}\PY{n}{beta} \PY{o}{*} \PY{n}{beta\PYZus{}dot}\PY{p}{)} \PY{o}{*}\PY{o}{*} \PY{l+m+mi}{2}\PY{p}{)}\PY{o}{/}\PY{p}{(}\PY{l+m+mi}{1} \PY{o}{\PYZhy{}} \PY{p}{(}\PY{n}{beta} \PY{o}{*}\PY{o}{*} \PY{l+m+mi}{2}\PY{p}{)}\PY{p}{)} 
  \PY{n}{power} \PY{o}{=} \PY{p}{(}\PY{n}{charge} \PY{o}{*}\PY{o}{*} \PY{l+m+mi}{2}\PY{p}{)} \PY{o}{*} \PY{p}{(}\PY{n}{gamma} \PY{o}{*}\PY{o}{*} \PY{l+m+mi}{4}\PY{p}{)} \PY{o}{*} \PY{n}{beta\PYZus{}term} \PY{o}{/} \PY{p}{(}\PY{l+m+mi}{6} \PY{o}{*} 
                  \PY{n+nb}{float}\PY{p}{(}\PY{n}{sym}\PY{o}{.}\PY{n}{N}\PY{p}{(}\PY{n}{sym}\PY{o}{.}\PY{n}{pi}\PY{p}{)}\PY{p}{)} \PY{o}{*} \PY{n}{constants}\PY{o}{.}\PY{n}{eps0} \PY{o}{*} \PY{n}{constants}\PY{o}{.}\PY{n}{c}\PY{p}{)}
  \PY{k}{return} \PY{n}{power}\PY{o}{.}\PY{n}{to}\PY{p}{(}\PY{n}{units}\PY{o}{.}\PY{n}{W}\PY{p}{)}
\end{Verbatim}
\end{tcolorbox}

    \begin{tcolorbox}[breakable, size=fbox, boxrule=1pt, pad at break*=1mm,colback=cellbackground, colframe=cellborder]
\prompt{In}{incolor}{181}{\boxspacing}
\begin{Verbatim}[commandchars=\\\{\}]
\PY{n}{total\PYZus{}brehmsstrahlung\PYZus{}power}\PY{p}{(}\PY{o}{.}\PY{l+m+mi}{0000001} \PY{o}{*} \PY{n}{constants}\PY{o}{.}\PY{n}{c}\PY{p}{,} \PY{l+m+mi}{1} \PY{o}{*} \PY{n}{constants}\PY{o}{.}\PY{n}{e}\PY{o}{.}\PY{n}{si}\PY{p}{,} \PY{o}{.}\PY{l+m+mi}{0000001} \PY{o}{*} 
                          \PY{n}{constants}\PY{o}{.}\PY{n}{c} \PY{o}{/} \PY{n}{units}\PY{o}{.}\PY{n}{s}\PY{p}{)}
\end{Verbatim}
\end{tcolorbox}
 
            
\prompt{Out}{outcolor}{181}{}
    
    $5.1303882 \times 10^{-51} \; \mathrm{W}$

    

    \begin{tcolorbox}[breakable, size=fbox, boxrule=1pt, pad at break*=1mm,colback=cellbackground, colframe=cellborder]
\prompt{In}{incolor}{182}{\boxspacing}
\begin{Verbatim}[commandchars=\\\{\}]
\PY{n}{total\PYZus{}energy\PYZus{}stored} \PY{o}{=} \PY{n}{half\PYZus{}life\PYZus{}Ni\PYZus{}63} \PY{o}{*} \PY{n}{units}\PY{o}{.}\PY{n}{s} \PY{o}{*} \PY{n}{power\PYZus{}Ni\PYZus{}63\PYZus{}per\PYZus{}gram}
\PY{n}{total\PYZus{}energy\PYZus{}stored}\PY{o}{.}\PY{n}{to}\PY{p}{(}\PY{n}{units}\PY{o}{.}\PY{n}{TW} \PY{o}{*} \PY{n}{units}\PY{o}{.}\PY{n}{hour}\PY{p}{)} \PY{c+c1}{\PYZsh{}underestimate, and also per gram}
\end{Verbatim}
\end{tcolorbox}
 
            
\prompt{Out}{outcolor}{182}{}
    
    $2.5064712 \times 10^{-9} \; \mathrm{TW\,h}$

    

    \#Using the technology of the Andasol Solar Power Station, an area half
the size of Pennsylvania could power the entire world.

    \begin{tcolorbox}[breakable, size=fbox, boxrule=1pt, pad at break*=1mm,colback=cellbackground, colframe=cellborder]
\prompt{In}{incolor}{183}{\boxspacing}
\begin{Verbatim}[commandchars=\\\{\}]
\PY{c+c1}{\PYZsh{}https://en.wikipedia.org/wiki/Andasol\PYZus{}Solar\PYZus{}Power\PYZus{}Station}
\PY{n}{andasol} \PY{o}{=} \PY{l+m+mi}{2000} \PY{o}{*} \PY{n}{units}\PY{o}{.}\PY{n}{kW} \PY{o}{*} \PY{n}{units}\PY{o}{.}\PY{n}{hour} \PY{o}{/} \PY{p}{(}\PY{p}{(}\PY{n}{units}\PY{o}{.}\PY{n}{m} \PY{o}{*}\PY{o}{*} \PY{l+m+mi}{2} \PY{p}{)} \PY{o}{*} \PY{n}{units}\PY{o}{.}\PY{n}{year}\PY{p}{)}
\PY{n}{andasol\PYZus{}efficiency} \PY{o}{=} \PY{n}{andasol}\PY{o}{.}\PY{n}{to}\PY{p}{(}\PY{n}{units}\PY{o}{.}\PY{n}{W} \PY{o}{/} \PY{p}{(}\PY{n}{units}\PY{o}{.}\PY{n}{m} \PY{o}{*}\PY{o}{*} \PY{l+m+mi}{2}\PY{p}{)}\PY{p}{)}
\PY{n}{andasol\PYZus{}efficiency}
\end{Verbatim}
\end{tcolorbox}
 
            
\prompt{Out}{outcolor}{183}{}
    
    $228.15423 \; \mathrm{\frac{W}{m^{2}}}$

    

    \begin{tcolorbox}[breakable, size=fbox, boxrule=1pt, pad at break*=1mm,colback=cellbackground, colframe=cellborder]
\prompt{In}{incolor}{184}{\boxspacing}
\begin{Verbatim}[commandchars=\\\{\}]
\PY{n}{needed\PYZus{}area} \PY{o}{=} \PY{p}{(}\PY{n}{global\PYZus{}power\PYZus{}consumed} \PY{o}{/} \PY{n}{andasol\PYZus{}efficiency}\PY{p}{)}\PY{o}{.}\PY{n}{to}\PY{p}{(}\PY{n}{units}\PY{o}{.}\PY{n}{km} \PY{o}{*}\PY{o}{*} \PY{l+m+mi}{2}\PY{p}{)}
\PY{n}{needed\PYZus{}area} 
\end{Verbatim}
\end{tcolorbox}
 
            
\prompt{Out}{outcolor}{184}{}
    
    $62500 \; \mathrm{km^{2}}$

    

    \begin{tcolorbox}[breakable, size=fbox, boxrule=1pt, pad at break*=1mm,colback=cellbackground, colframe=cellborder]
\prompt{In}{incolor}{185}{\boxspacing}
\begin{Verbatim}[commandchars=\\\{\}]
\PY{n}{pennsylvania\PYZus{}area} \PY{o}{=} \PY{l+m+mf}{119281.9} \PY{o}{*} \PY{p}{(}\PY{n}{units}\PY{o}{.}\PY{n}{km} \PY{o}{*}\PY{o}{*} \PY{l+m+mi}{2}\PY{p}{)}
\PY{n}{needed\PYZus{}area} \PY{o}{/} \PY{n}{pennsylvania\PYZus{}area}
\end{Verbatim}
\end{tcolorbox}
 
            
\prompt{Out}{outcolor}{185}{}
    
    $0.52396885 \; \mathrm{}$

    

    \begin{tcolorbox}[breakable, size=fbox, boxrule=1pt, pad at break*=1mm,colback=cellbackground, colframe=cellborder]
\prompt{In}{incolor}{186}{\boxspacing}
\begin{Verbatim}[commandchars=\\\{\}]
\PY{n}{neutrino\PYZus{}mass} \PY{o}{=} \PY{p}{(}\PY{l+m+mi}{1} \PY{o}{*} \PY{n}{units}\PY{o}{.}\PY{n}{keV} \PY{o}{/} \PY{p}{(}\PY{n}{constants}\PY{o}{.}\PY{n}{c} \PY{o}{*}\PY{o}{*} \PY{l+m+mi}{2}\PY{p}{)}\PY{p}{)}\PY{o}{.}\PY{n}{to}\PY{p}{(}\PY{n}{units}\PY{o}{.}\PY{n}{g}\PY{p}{)}
\PY{n}{neutrino\PYZus{}mass}
\end{Verbatim}
\end{tcolorbox}
 
            
\prompt{Out}{outcolor}{186}{}
    
    $1.7826619 \times 10^{-30} \; \mathrm{g}$

    

    \#Simualted Spectrum Using the Fermi Distribution
\href{https://photos.app.goo.gl/gfHG43iBdRp8a59v9}{Approximated Energy
Spectrum} \begin{equation}
\begin{split}
N(T_e) = \frac{C}{c^5}(Q - T_e)^2(T + m_ec^2)\sqrt{T_e^2 + 2T_em_ec^2}\\
\end{split}
\end{equation} Integrating this from 0 to Q and then normalizing the
distribution \begin{equation}
\begin{split}
\int_0^{T_{e \ max}} N(T_e)dT_e\\
\int_0^{T_{e \ max}} \frac{C}{c^5}(Q - T_e)^2(T_e + m_ec^2)\sqrt{T_e^2 + 2T_em_ec^2} dT_e = 1\\
C = \frac{c^5}{\int_0^{Q} (Q - x)^2(x + a)\sqrt{x^2 + 2a} dx}\\
\end{split}
\end{equation} For some reason,
\href{https://www.wolframalpha.com/input?i=\%5Cint_0\%5E\%7BQ\%7D+\%28Q+-+x\%29\%5E2\%28x+\%2B+a\%29\%5Csqrt\%7Bx\%5E2+\%2B+2a\%7D+dx}{WolframAlpha}
won't evaluate the integral. So using sympy, where Q = .782 MeV, a =
m\(_e\)c\(^2\), and \(x = T_e\) All units are in kg-m-s SI.

    \begin{tcolorbox}[breakable, size=fbox, boxrule=1pt, pad at break*=1mm,colback=cellbackground, colframe=cellborder]
\prompt{In}{incolor}{187}{\boxspacing}
\begin{Verbatim}[commandchars=\\\{\}]
\PY{n}{Q\PYZus{}fermi\PYZus{}distr} \PY{o}{=} \PY{n+nb}{float}\PY{p}{(}\PY{p}{(}\PY{o}{.}\PY{l+m+mi}{782} \PY{o}{*} \PY{n}{units}\PY{o}{.}\PY{n}{MeV}\PY{p}{)}\PY{o}{.}\PY{n}{to}\PY{p}{(}\PY{n}{units}\PY{o}{.}\PY{n}{J}\PY{p}{)} \PY{o}{/} \PY{n}{units}\PY{o}{.}\PY{n}{J}\PY{p}{)}
\PY{n}{Q\PYZus{}fermi\PYZus{}distr}
\end{Verbatim}
\end{tcolorbox}

            \begin{tcolorbox}[breakable, size=fbox, boxrule=.5pt, pad at break*=1mm, opacityfill=0]
\prompt{Out}{outcolor}{187}{\boxspacing}
\begin{Verbatim}[commandchars=\\\{\}]
1.252902127788e-13
\end{Verbatim}
\end{tcolorbox}
        
    \begin{tcolorbox}[breakable, size=fbox, boxrule=1pt, pad at break*=1mm,colback=cellbackground, colframe=cellborder]
\prompt{In}{incolor}{188}{\boxspacing}
\begin{Verbatim}[commandchars=\\\{\}]
\PY{n}{unnormalized\PYZus{}fermi\PYZus{}fun} \PY{o}{=} \PY{p}{(}\PY{p}{(}\PY{n}{Q} \PY{o}{\PYZhy{}} \PY{n}{x}\PY{p}{)} \PY{o}{*}\PY{o}{*} \PY{l+m+mi}{2}\PY{p}{)} \PY{o}{*} \PY{p}{(}\PY{n}{x} \PY{o}{+} \PY{n}{a}\PY{p}{)} \PY{o}{*} \PY{n}{sym}\PY{o}{.}\PY{n}{sqrt}\PY{p}{(}
    \PY{p}{(}\PY{n}{x} \PY{o}{*}\PY{o}{*} \PY{l+m+mi}{2}\PY{p}{)} \PY{o}{+} \PY{p}{(}\PY{l+m+mi}{2} \PY{o}{*} \PY{n}{a}\PY{p}{)}\PY{p}{)}
\PY{n}{denominator} \PY{o}{=} \PY{n}{sym}\PY{o}{.}\PY{n}{integrate}\PY{p}{(}\PY{n}{unnormalized\PYZus{}fermi\PYZus{}fun}\PY{p}{,} \PY{p}{(}\PY{n}{x}\PY{p}{,} \PY{l+m+mi}{0}\PY{p}{,} \PY{n}{sym}\PY{o}{.}\PY{n}{oo}\PY{p}{)}\PY{p}{)}
\end{Verbatim}
\end{tcolorbox}

    \begin{tcolorbox}[breakable, size=fbox, boxrule=1pt, pad at break*=1mm,colback=cellbackground, colframe=cellborder]
\prompt{In}{incolor}{189}{\boxspacing}
\begin{Verbatim}[commandchars=\\\{\}]
\PY{n}{a\PYZus{}fermi\PYZus{}distr} \PY{o}{=} \PY{n+nb}{float}\PY{p}{(}\PY{p}{(}\PY{n}{constants}\PY{o}{.}\PY{n}{m\PYZus{}e} \PY{o}{*} \PY{p}{(}\PY{n}{constants}\PY{o}{.}\PY{n}{c} \PY{o}{*}\PY{o}{*} \PY{l+m+mi}{2}\PY{p}{)}\PY{p}{)}\PY{o}{.}\PY{n}{to}\PY{p}{(}\PY{n}{units}\PY{o}{.}\PY{n}{J}\PY{p}{)} 
\PY{o}{/} \PY{n}{units}\PY{o}{.}\PY{n}{J}\PY{p}{)}
\PY{n}{a\PYZus{}fermi\PYZus{}distr}
\end{Verbatim}
\end{tcolorbox}

            \begin{tcolorbox}[breakable, size=fbox, boxrule=.5pt, pad at break*=1mm, opacityfill=0]
\prompt{Out}{outcolor}{189}{\boxspacing}
\begin{Verbatim}[commandchars=\\\{\}]
8.187105776823886e-14
\end{Verbatim}
\end{tcolorbox}
        
    \begin{tcolorbox}[breakable, size=fbox, boxrule=1pt, pad at break*=1mm,colback=cellbackground, colframe=cellborder]
\prompt{In}{incolor}{190}{\boxspacing}
\begin{Verbatim}[commandchars=\\\{\}]
\PY{n}{c\PYZus{}to\PYZus{}the\PYZus{}fifth} \PY{o}{=} \PY{n+nb}{float}\PY{p}{(}\PY{p}{(}\PY{n}{constants}\PY{o}{.}\PY{n}{c} \PY{o}{*} \PY{n}{units}\PY{o}{.}\PY{n}{s} \PY{o}{/} \PY{n}{units}\PY{o}{.}\PY{n}{m}\PY{p}{)} \PY{o}{*}\PY{o}{*} \PY{l+m+mi}{5}\PY{p}{)}
\PY{n}{c\PYZus{}to\PYZus{}the\PYZus{}fifth}
\end{Verbatim}
\end{tcolorbox}

            \begin{tcolorbox}[breakable, size=fbox, boxrule=.5pt, pad at break*=1mm, opacityfill=0]
\prompt{Out}{outcolor}{190}{\boxspacing}
\begin{Verbatim}[commandchars=\\\{\}]
2.4216061708512208e+42
\end{Verbatim}
\end{tcolorbox}
        
    \begin{tcolorbox}[breakable, size=fbox, boxrule=1pt, pad at break*=1mm,colback=cellbackground, colframe=cellborder]
\prompt{In}{incolor}{191}{\boxspacing}
\begin{Verbatim}[commandchars=\\\{\}]
\PY{n}{integrated\PYZus{}dist} \PY{o}{=} \PY{n}{sym}\PY{o}{.}\PY{n}{lambdify}\PY{p}{(}\PY{p}{(}\PY{n}{Q}\PY{p}{,} \PY{n}{a}\PY{p}{)}\PY{p}{,} \PY{n}{denominator}\PY{p}{)}
\PY{n}{fermi\PYZus{}distr\PYZus{}const} \PY{o}{=} \PY{n}{c\PYZus{}to\PYZus{}the\PYZus{}fifth} \PY{o}{/} \PY{n}{integrated\PYZus{}dist}\PY{p}{(}\PY{n}{Q\PYZus{}fermi\PYZus{}distr}\PY{p}{,} 
                                                     \PY{n}{a\PYZus{}fermi\PYZus{}distr}\PY{p}{)}
\end{Verbatim}
\end{tcolorbox}

    \begin{tcolorbox}[breakable, size=fbox, boxrule=1pt, pad at break*=1mm,colback=cellbackground, colframe=cellborder]
\prompt{In}{incolor}{192}{\boxspacing}
\begin{Verbatim}[commandchars=\\\{\}]
\PY{n}{fermi\PYZus{}distr\PYZus{}fun} \PY{o}{=} \PY{n}{C} \PY{o}{*} \PY{n}{unnormalized\PYZus{}fermi\PYZus{}fun} \PY{o}{/} \PY{n}{c\PYZus{}to\PYZus{}the\PYZus{}fifth}
\PY{n}{fermi\PYZus{}distr} \PY{o}{=} \PY{n}{sym}\PY{o}{.}\PY{n}{lambdify}\PY{p}{(}\PY{p}{(}\PY{n}{Q}\PY{p}{,} \PY{n}{a}\PY{p}{,} \PY{n}{C}\PY{p}{,} \PY{n}{x}\PY{p}{)}\PY{p}{,} \PY{n}{fermi\PYZus{}distr\PYZus{}fun}\PY{p}{)}

\PY{n}{electron\PYZus{}energies} \PY{o}{=} \PY{p}{[}\PY{p}{]}
\PY{n}{abundance} \PY{o}{=} \PY{p}{[}\PY{p}{]}
\PY{n}{log\PYZus{}res} \PY{o}{=} \PY{l+m+mi}{3}
\PY{k}{for} \PY{n}{electron\PYZus{}energy} \PY{o+ow}{in} \PY{n}{np}\PY{o}{.}\PY{n}{linspace}\PY{p}{(}\PY{l+m+mi}{0}\PY{p}{,} \PY{n}{Q\PYZus{}fermi\PYZus{}distr}\PY{p}{,} \PY{l+m+mi}{10} \PY{o}{*}\PY{o}{*} \PY{n}{log\PYZus{}res}\PY{p}{)}\PY{p}{:}
  \PY{n}{abundance}\PY{o}{.}\PY{n}{append}\PY{p}{(}\PY{n}{fermi\PYZus{}distr}\PY{p}{(}\PY{n}{Q\PYZus{}fermi\PYZus{}distr}\PY{p}{,} \PY{n}{a\PYZus{}fermi\PYZus{}distr}\PY{p}{,} 
                               \PY{n}{fermi\PYZus{}distr\PYZus{}const}\PY{p}{,} \PY{n}{electron\PYZus{}energy}\PY{p}{)}\PY{p}{)}
  \PY{n}{electron\PYZus{}energy\PYZus{}keV} \PY{o}{=} \PY{n+nb}{float}\PY{p}{(}\PY{p}{(}\PY{n}{electron\PYZus{}energy} \PY{o}{*} \PY{n}{units}\PY{o}{.}\PY{n}{J}\PY{p}{)}\PY{o}{.}\PY{n}{to}\PY{p}{(}\PY{n}{units}\PY{o}{.}\PY{n}{keV}\PY{p}{)} 
        \PY{o}{/} \PY{n}{units}\PY{o}{.}\PY{n}{keV}\PY{p}{)}
  \PY{n}{electron\PYZus{}energies}\PY{o}{.}\PY{n}{append}\PY{p}{(}\PY{n}{electron\PYZus{}energy\PYZus{}keV}\PY{p}{)}
\PY{n}{xlabel}\PY{p}{,} \PY{n}{ylabel} \PY{o}{=} \PY{l+s+s1}{\PYZsq{}}\PY{l+s+s1}{Electron Energy in keV}\PY{l+s+s1}{\PYZsq{}}\PY{p}{,} \PY{l+s+s1}{\PYZsq{}}\PY{l+s+s1}{Abundance in Counts}\PY{l+s+s1}{\PYZsq{}} 
\PY{n}{data} \PY{o}{=} \PY{p}{\PYZob{}}\PY{n}{xlabel}\PY{p}{:} \PY{n}{electron\PYZus{}energies}\PY{p}{,} \PY{n}{ylabel}\PY{p}{:} \PY{n}{abundance}\PY{p}{\PYZcb{}}
\PY{n}{Ni63\PYZus{}spectrum} \PY{o}{=} \PY{n}{pd}\PY{o}{.}\PY{n}{DataFrame}\PY{o}{.}\PY{n}{from\PYZus{}dict}\PY{p}{(}\PY{n}{data}\PY{p}{)}
\PY{n}{fig} \PY{o}{=} \PY{n}{px}\PY{o}{.}\PY{n}{scatter}\PY{p}{(}\PY{n}{Ni63\PYZus{}spectrum}\PY{p}{,} \PY{n}{x} \PY{o}{=} \PY{n}{xlabel}\PY{p}{,} \PY{n}{y} \PY{o}{=} \PY{n}{ylabel}\PY{p}{)}
\PY{n}{fig}\PY{o}{.}\PY{n}{show}\PY{p}{(}\PY{p}{)}
\end{Verbatim}
\end{tcolorbox}

    
    
    \begin{tcolorbox}[breakable, size=fbox, boxrule=1pt, pad at break*=1mm,colback=cellbackground, colframe=cellborder]
\prompt{In}{incolor}{193}{\boxspacing}
\begin{Verbatim}[commandchars=\\\{\}]
\PY{n}{fermi\PYZus{}distr\PYZus{}fun}
\end{Verbatim}
\end{tcolorbox}
 
 \begin{center}
    \adjustimage{max size={0.9\linewidth}{0.9\paperheight}}{Images/plotly.PNG}
    \end{center}
    { \hspace*{\fill} \\}
            
\prompt{Out}{outcolor}{193}{}
    
    $\displaystyle 4.12949063327043 \cdot 10^{-43} C \left(Q - x\right)^{2} \left(a + x\right) \sqrt{2 a + x^{2}}$

    

    \begin{tcolorbox}[breakable, size=fbox, boxrule=1pt, pad at break*=1mm,colback=cellbackground, colframe=cellborder]
\prompt{In}{incolor}{194}{\boxspacing}
\begin{Verbatim}[commandchars=\\\{\}]
\PY{n}{average\PYZus{}fun} \PY{o}{=} \PY{p}{(}\PY{n}{sym}\PY{o}{.}\PY{n}{integrate}\PY{p}{(}\PY{n}{fermi\PYZus{}distr\PYZus{}fun}\PY{p}{,} \PY{p}{(}\PY{n}{x}\PY{p}{,} \PY{l+m+mi}{0}\PY{p}{,} \PY{n}{fermi\PYZus{}distr\PYZus{}fun}\PY{p}{)}\PY{p}{)} \PY{o}{/} 
           \PY{n}{fermi\PYZus{}distr\PYZus{}fun}\PY{p}{)}
\PY{n}{average} \PY{o}{=} \PY{n}{sym}\PY{o}{.}\PY{n}{lambdify}\PY{p}{(}\PY{p}{(}\PY{n}{Q}\PY{p}{,} \PY{n}{a}\PY{p}{,} \PY{n}{C}\PY{p}{,} \PY{n}{x}\PY{p}{)}\PY{p}{,} \PY{n}{average\PYZus{}fun}\PY{p}{)}
\PY{n}{average}\PY{p}{(}\PY{n}{Q\PYZus{}fermi\PYZus{}distr}\PY{p}{,} \PY{n}{a\PYZus{}fermi\PYZus{}distr}\PY{p}{,} \PY{n}{fermi\PYZus{}distr\PYZus{}const}\PY{p}{,} \PY{n}{Q\PYZus{}fermi\PYZus{}distr}\PY{p}{)}
\end{Verbatim}
\end{tcolorbox}

    \begin{Verbatim}[commandchars=\\\{\}]
<lambdifygenerated-7>:2: RuntimeWarning:

invalid value encountered in double\_scalars

    \end{Verbatim}

            \begin{tcolorbox}[breakable, size=fbox, boxrule=.5pt, pad at break*=1mm, opacityfill=0]
\prompt{Out}{outcolor}{194}{\boxspacing}
\begin{Verbatim}[commandchars=\\\{\}]
nan
\end{Verbatim}
\end{tcolorbox}
        
    \begin{tcolorbox}[breakable, size=fbox, boxrule=1pt, pad at break*=1mm,colback=cellbackground, colframe=cellborder]
\prompt{In}{incolor}{195}{\boxspacing}
\begin{Verbatim}[commandchars=\\\{\}]
\PY{k}{def} \PY{n+nf}{normalized\PYZus{}counts}\PY{p}{(}\PY{n}{T\PYZus{}e}\PY{p}{,} \PY{n}{Q}\PY{p}{)}\PY{p}{:}
  \PY{n}{count} \PY{o}{=} \PY{p}{(}\PY{n}{np}\PY{o}{.}\PY{n}{sqrt}\PY{p}{(}\PY{p}{(}\PY{p}{(}\PY{n}{T\PYZus{}e} \PY{o}{*}\PY{o}{*}  \PY{l+m+mi}{2}\PY{p}{)} \PY{o}{+} \PY{l+m+mi}{2} \PY{o}{*} \PY{p}{(}\PY{n}{T\PYZus{}e} \PY{o}{*} \PY{n}{m\PYZus{}e} \PY{o}{*} \PY{p}{(}\PY{n}{c} \PY{o}{*}\PY{o}{*} \PY{l+m+mi}{2}\PY{p}{)}\PY{p}{)}\PY{p}{)}\PY{p}{)} 
            \PY{o}{*} \PY{p}{(}\PY{p}{(}\PY{n}{Q} \PY{o}{\PYZhy{}} \PY{n}{T\PYZus{}e}\PY{p}{)} \PY{o}{*}\PY{o}{*} \PY{l+m+mi}{2}\PY{p}{)} \PY{o}{*} \PY{p}{(}\PY{n}{T\PYZus{}e} \PY{o}{+} \PY{n}{m\PYZus{}e} \PY{o}{*} \PY{p}{(}\PY{n}{c} \PY{o}{*}\PY{o}{*} \PY{l+m+mi}{2}\PY{p}{)}\PY{p}{)} \PY{o}{/} \PY{p}{(}\PY{n}{c} \PY{o}{*}\PY{o}{*} \PY{l+m+mi}{5}\PY{p}{)}\PY{p}{)}
  \PY{k}{return} \PY{n}{count}
\end{Verbatim}
\end{tcolorbox}

    \begin{tcolorbox}[breakable, size=fbox, boxrule=1pt, pad at break*=1mm,colback=cellbackground, colframe=cellborder]
\prompt{In}{incolor}{196}{\boxspacing}
\begin{Verbatim}[commandchars=\\\{\}]
\PY{n}{fermi\PYZus{}distr\PYZus{}const}
\end{Verbatim}
\end{tcolorbox}

            \begin{tcolorbox}[breakable, size=fbox, boxrule=.5pt, pad at break*=1mm, opacityfill=0]
\prompt{Out}{outcolor}{196}{\boxspacing}
\begin{Verbatim}[commandchars=\\\{\}]
0.0
\end{Verbatim}
\end{tcolorbox}
        
    \begin{tcolorbox}[breakable, size=fbox, boxrule=1pt, pad at break*=1mm,colback=cellbackground, colframe=cellborder]
\prompt{In}{incolor}{197}{\boxspacing}
\begin{Verbatim}[commandchars=\\\{\}]
\PY{l+s+sd}{\PYZsq{}\PYZsq{}\PYZsq{}}
\PY{l+s+sd}{We need to account for the fact that not all of the electrons will have the }
\PY{l+s+sd}{maximum energy (Q value)!}
\PY{l+s+sd}{stuck for right now. Use}
\PY{l+s+sd}{https://github.com/MarcosP7635/Computing\PYZhy{}and\PYZhy{}Formatting/blob/main/error\PYZus{}propagation.py}
\PY{l+s+sd}{as a sympy reference}
\PY{l+s+sd}{They average out to 17 keV. Rewrite to draw from a database of beta emission }
\PY{l+s+sd}{spectra. }
\PY{l+s+sd}{\PYZsq{}\PYZsq{}\PYZsq{}}
\end{Verbatim}
\end{tcolorbox}

            \begin{tcolorbox}[breakable, size=fbox, boxrule=.5pt, pad at break*=1mm, opacityfill=0]
\prompt{Out}{outcolor}{197}{\boxspacing}
\begin{Verbatim}[commandchars=\\\{\}]
'\textbackslash{}nWe need to account for the fact that not all of the electrons will have the
\textbackslash{}nmaximum energy (Q value)!\textbackslash{}nstuck for right now.
Use\textbackslash{}nhttps://github.com/MarcosP7635/Computing-and-
Formatting/blob/main/error\_propagation.py\textbackslash{}nas a sympy reference\textbackslash{}nThey average
out to 17 keV. Rewrite to draw from a database of beta emission \textbackslash{}nspectra. \textbackslash{}n'
\end{Verbatim}
\end{tcolorbox}
        
    \begin{tcolorbox}[breakable, size=fbox, boxrule=1pt, pad at break*=1mm,colback=cellbackground, colframe=cellborder]
\prompt{In}{incolor}{198}{\boxspacing}
\begin{Verbatim}[commandchars=\\\{\}]
\PY{n}{global\PYZus{}power\PYZus{}consumed}
\end{Verbatim}
\end{tcolorbox}
 
            
\prompt{Out}{outcolor}{198}{}
    
    $1.425964 \times 10^{13} \; \mathrm{W}$

    

    \begin{tcolorbox}[breakable, size=fbox, boxrule=1pt, pad at break*=1mm,colback=cellbackground, colframe=cellborder]
\prompt{In}{incolor}{199}{\boxspacing}
\begin{Verbatim}[commandchars=\\\{\}]
\PY{n}{energy\PYZus{}Ni\PYZus{}63\PYZus{}per\PYZus{}gram} \PY{o}{=} \PY{p}{(}\PY{n}{power\PYZus{}Ni\PYZus{}63\PYZus{}per\PYZus{}gram} \PY{o}{*} \PY{l+m+mi}{50} \PY{o}{*} \PY{n}{units}\PY{o}{.}\PY{n}{year}\PY{p}{)}\PY{o}{.}\PY{n}{to}\PY{p}{(}\PY{n}{units}\PY{o}{.}\PY{n}{J}\PY{p}{)}
\PY{l+s+sd}{\PYZsq{}\PYZsq{}\PYZsq{}}
\PY{l+s+sd}{The best lithium\PYZhy{}ion batteries store less than 1 kJ/g}
\PY{l+s+sd}{Source: https://doi.org/10.1039/D0EE02681F }
\PY{l+s+sd}{\PYZsq{}\PYZsq{}\PYZsq{}}
\PY{n}{energy\PYZus{}Ni\PYZus{}63\PYZus{}per\PYZus{}gram} 
\end{Verbatim}
\end{tcolorbox}
 
            
\prompt{Out}{outcolor}{199}{}
    
    $4507141 \; \mathrm{J}$

    

    \begin{tcolorbox}[breakable, size=fbox, boxrule=1pt, pad at break*=1mm,colback=cellbackground, colframe=cellborder]
\prompt{In}{incolor}{200}{\boxspacing}
\begin{Verbatim}[commandchars=\\\{\}]
\PY{l+s+sd}{\PYZsq{}\PYZsq{}\PYZsq{}}
\PY{l+s+sd}{if google drive won\PYZsq{}t mount to the colab session, then}
\PY{l+s+sd}{you need to download this current python notebook and upload}
\PY{l+s+sd}{it to the colab session, then right click on it to copy the}
\PY{l+s+sd}{path for the command below.}
\PY{l+s+sd}{\PYZsq{}\PYZsq{}\PYZsq{}}
\PY{o}{!}jupyter nbconvert \PYZhy{}\PYZhy{}to LaTeX /Energy.ipynb
\PY{c+c1}{\PYZsh{}The above line makes a .tex file to format this Jupyter Notebook}
\end{Verbatim}
\end{tcolorbox}

    \begin{Verbatim}[commandchars=\\\{\}]
This application is used to convert notebook files (*.ipynb) to various other
formats.

WARNING: THE COMMANDLINE INTERFACE MAY CHANGE IN FUTURE RELEASES.

Options

-------


    \end{Verbatim}

    \begin{Verbatim}[commandchars=\\\{\}]
[NbConvertApp] WARNING | pattern '/Energy.ipynb' matched no files
    \end{Verbatim}

    \begin{Verbatim}[commandchars=\\\{\}]

Arguments that take values are actually convenience aliases to full
Configurables, whose aliases are listed on the help line. For more information
on full configurables, see '--help-all'.


--debug

    set log level to logging.DEBUG (maximize logging output)

--generate-config

    generate default config file

-y

    Answer yes to any questions instead of prompting.

--execute

    Execute the notebook prior to export.

--allow-errors

    Continue notebook execution even if one of the cells throws an error and
include the error message in the cell output (the default behaviour is to abort
conversion). This flag is only relevant if '--execute' was specified, too.

--stdin

    read a single notebook file from stdin. Write the resulting notebook with
default basename 'notebook.*'

--stdout

    Write notebook output to stdout instead of files.

--inplace

    Run nbconvert in place, overwriting the existing notebook (only
    relevant when converting to notebook format)

--clear-output

    Clear output of current file and save in place,
    overwriting the existing notebook.

--no-prompt

    Exclude input and output prompts from converted document.

--no-input

    Exclude input cells and output prompts from converted document.
    This mode is ideal for generating code-free reports.
--log-level=<Enum> (Application.log\_level)

    Default: 30

    Choices: (0, 10, 20, 30, 40, 50, 'DEBUG', 'INFO', 'WARN', 'ERROR',
'CRITICAL')

    Set the log level by value or name.

--config=<Unicode> (JupyterApp.config\_file)

    Default: ''

    Full path of a config file.

--to=<Unicode> (NbConvertApp.export\_format)

    Default: 'html'

    The export format to be used, either one of the built-in formats

    ['asciidoc', 'custom', 'html', 'html\_ch', 'html\_embed', 'html\_toc',

    'html\_with\_lenvs', 'html\_with\_toclenvs', 'latex', 'latex\_with\_lenvs',

    'markdown', 'notebook', 'pdf', 'python', 'rst', 'script', 'selectLanguage',

    'slides', 'slides\_with\_lenvs'] or a dotted object name that represents the

    import path for an `Exporter` class

--template=<Unicode> (TemplateExporter.template\_file)

    Default: ''

    Name of the template file to use

--writer=<DottedObjectName> (NbConvertApp.writer\_class)

    Default: 'FilesWriter'

    Writer class used to write the  results of the conversion

--post=<DottedOrNone> (NbConvertApp.postprocessor\_class)

    Default: ''

    PostProcessor class used to write the results of the conversion

--output=<Unicode> (NbConvertApp.output\_base)

    Default: ''

    overwrite base name use for output files. can only be used when converting

    one notebook at a time.

--output-dir=<Unicode> (FilesWriter.build\_directory)

    Default: ''

    Directory to write output(s) to. Defaults to output to the directory of each

    notebook. To recover previous default behaviour (outputting to the current

    working directory) use . as the flag value.

--reveal-prefix=<Unicode> (SlidesExporter.reveal\_url\_prefix)

    Default: ''

    The URL prefix for reveal.js (version 3.x). This defaults to the reveal CDN,

    but can be any url pointing to a copy  of reveal.js.

    For speaker notes to work, this must be a relative path to a local  copy of

    reveal.js: e.g., "reveal.js".

    If a relative path is given, it must be a subdirectory of the current

    directory (from which the server is run).

    See the usage documentation

    (https://nbconvert.readthedocs.io/en/latest/usage.html\#reveal-js-html-

    slideshow) for more details.

--nbformat=<Enum> (NotebookExporter.nbformat\_version)

    Default: 4

    Choices: [1, 2, 3, 4]

    The nbformat version to write. Use this to downgrade notebooks.

To see all available configurables, use `--help-all`

Examples
--------

    The simplest way to use nbconvert is

    > jupyter nbconvert mynotebook.ipynb

    which will convert mynotebook.ipynb to the default format (probably HTML).

    You can specify the export format with `--to`.
    Options include ['asciidoc', 'custom', 'html', 'html\_ch', 'html\_embed',
'html\_toc', 'html\_with\_lenvs', 'html\_with\_toclenvs', 'latex',
'latex\_with\_lenvs', 'markdown', 'notebook', 'pdf', 'python', 'rst', 'script',
'selectLanguage', 'slides', 'slides\_with\_lenvs'].

    > jupyter nbconvert --to latex mynotebook.ipynb

    Both HTML and LaTeX support multiple output templates. LaTeX includes
    'base', 'article' and 'report'.  HTML includes 'basic' and 'full'. You
    can specify the flavor of the format used.

    > jupyter nbconvert --to html --template basic mynotebook.ipynb

    You can also pipe the output to stdout, rather than a file

    > jupyter nbconvert mynotebook.ipynb --stdout

    PDF is generated via latex

    > jupyter nbconvert mynotebook.ipynb --to pdf

    You can get (and serve) a Reveal.js-powered slideshow

    > jupyter nbconvert myslides.ipynb --to slides --post serve

    Multiple notebooks can be given at the command line in a couple of
    different ways:

    > jupyter nbconvert notebook*.ipynb
    > jupyter nbconvert notebook1.ipynb notebook2.ipynb

    or you can specify the notebooks list in a config file, containing::

        c.NbConvertApp.notebooks = ["my\_notebook.ipynb"]

    > jupyter nbconvert --config mycfg.py

    \end{Verbatim}

    \begin{tcolorbox}[breakable, size=fbox, boxrule=1pt, pad at break*=1mm,colback=cellbackground, colframe=cellborder]
\prompt{In}{incolor}{ }{\boxspacing}
\begin{Verbatim}[commandchars=\\\{\}]

\end{Verbatim}
\end{tcolorbox}


    % Add a bibliography block to the postdoc
    
    
    
\end{document}
